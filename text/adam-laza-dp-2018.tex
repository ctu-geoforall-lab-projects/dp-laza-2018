\documentclass[12pt,a4paper]{article}
\usepackage[utf8]{inputenc}
\usepackage[czech, english]{babel}
\usepackage[T1]{fontenc}
\usepackage{amsmath}
\usepackage{amsfonts}
\usepackage{amssymb}
\usepackage{graphicx}
\usepackage[final,pdftex, colorlinks=false]{hyperref}
\usepackage{xcolor}
\usepackage{comment}
\usepackage{todonotes}
\usepackage{floatrow}
\usepackage{multirow}
\usepackage{algorithm}
\usepackage{algorithmicx}
\usepackage{algpseudocode}
\usepackage{titletoc}
\usepackage{pdfpages}
\usepackage{hhline}
\usepackage{makecell}

\usepackage{listings}			%vkladani kodu
\lstset{basicstyle=\ttfamily,
  showstringspaces=false,
  commentstyle=\color{red},
  keywordstyle=\color{blue},
  breaklines=true,
  frame=lines,
}

%okraje
\usepackage[
left=35mm,
right=25mm,
top=40mm,
bottom=35mm]
{geometry}

\author{Adam Laža}

%%%%%%%%%%Prikazy%%%%%%%%%%
\renewcommand\baselinestretch{1.3}		%radkovani
\parskip=0.8ex plus 0.4ex minus 0.1 ex	%mezera mezi odstavci

\newcommand{\keywords}[2]{\noindent\textbf{#1: }#2}
\newcommand{\necislovana}[1]{%
\phantomsection
\addcontentsline{toc}{section}{#1}

%\newcommand{\exedout}{%
%  \rule{0.8\textwidth}{0.5\textwidth}%
%}


\section*{#1}
\markboth{\uppercase{#1}}{}
}
%%%%%%%%%%%%%%%%%%%%%%%%%%%%

%%%%%%%%%%Zahlavi%%%%%%%%%%%
\usepackage{fancyhdr}
\fancyhead[L]{CTU in~Prague}
\setlength{\headheight}{16pt}
%%%%%%%%%%%%%%%%%%%%%%%%%%%%

\begin{document}
\pagestyle{empty}

\newpage
\begin{center}
%napisy
\newcommand{\napisCVUT}{České vysoké učení technické v Praze}
\newcommand{\napisFS}{Fakulta stavební}
\newcommand{\napisProgram}{Studijní program Geodézie a kartografie}
\newcommand{\napisObor}{Obor Geomatika}
\newcommand{\napisKatedra}{Katedra geomatiky}
\newcommand{\napisVedouci}{Vedoucí práce: Ing. Martin Landa, Ph.D.}
\newcommand{\napisAutor}{Adam Laža}
\newcommand{\napisDatum}{Praha 2018}
\newcommand{\napisNazevI}{Nazev}
\newcommand{\napisNazevAjI}{Name}
\newcommand{\napisDiplomka}{Diplomová práce}
\newcommand{\napisPraha}{Praha 2018}
%
% prikazy
%\newcommand{\velka}[1]{\uppercase{#1}}
\newcommand{\velka}[1]{\textsc{#1}}
%
% 
\newif\ifpatitul
\patitultrue

\ifpatitul
{\Large\velka{\napisCVUT}}\\
\velka{\Large\napisFS}\\
\vfill
{\LARGE\velka{\napisDiplomka}}
\vfill
{\large\napisPraha\hfill\napisAutor}
\newpage
\fi%patitul


{\Large\velka{\napisCVUT}}\\
{\Large\velka{\napisFS}}\\
{\Large\velka{\napisProgram}}\\
{\Large\velka{\napisObor}}
\vfill
\includegraphics[width=3cm]{logo_cvut_cb} %~
\vfill
{\Large\velka{\napisDiplomka}}\\
\Large\velka{\napisNazevI}\\
\large\velka{\napisNazevAjI}
\vfill
{\large%
\napisVedouci\\
\napisKatedra\\
\bigskip
\napisDatum\hfill\napisAutor}
\end{center}


%\newpage
%\includepdf[pages={1}]{../formulare/zadani.pdf}

\newpage

\selectlanguage{english}
\begin{abstract}
\bigskip
\todo[inline]{Upravit abstract, aby odpovidal skutec strukture}
This master thesis is dedicated to an isolation of PyWPS processes as one of the OGC WPS implementation. OGC WPS is Web Processing Service
Standard defined by Open Geospatial Consortium. The practical part contains an introductory research where various solutions how 
to reach the process isolation are considered and described. Based on the research the Docker technology has been chosen for the 
implementation of process isolation.  In the theoretical part Docker technology is described as well as the OGC WPS standard and 
its PyWPS implementation written in Python.

\bigskip
\keywords{Keywords}
OGC WPS, PyWPS, Docker container, Python, process izolation, Web Processing Service.
\end{abstract}

\selectlanguage{czech}
\begin{abstract}
\bigskip
\todo[inline]{Překlad WPS - Webová Procesingová??? Služba, OGC??}
Tato diplomová práce se věnuje možnostem izolace procesů v rámci frameworku PyWPS jako jedné z implementací OGC WPS. Webová Procesingová 
Služba je standard vydaný a dále rozšiřovaný Open Geospatial Consorciem. Praktická část obsahuje úvodní rešerši, ve které jsou popsány
různé možnosti, jak izolace jednotlivých procesů dosáhnout. Na základě rešerše byla pro implementaci vybraná technologie Docker.
V teoretické části je popsána jak technologie Docker, tak OGC WPS standard a jeho implementace PyWPS napsaná v jazyce Python.

\bigskip
\keywords{Klíčová slova}
OGC WPS, PyWPS, Docker kontejner, Python, izolace procesu, Webová Procesingová Služba.
\end{abstract}

\selectlanguage{english}
%%%%Prohlaseni a podekovani
\newcommand{\odsaditodzhora}{\hskip1pt\vfill}
\newpage
\odsaditodzhora
\noindent {\bf Declaration of authorship}
I declare that the work presented here is, to the best of my knowledge and
belief, original and the result of my own investigations, except as acknowledged.
Formulations and ideas taken from other sources are cited as such.


\begin{flushleft}
\begin{tabular}{cp{0.3\textwidth}c}
In Prague .................
& 
&
..................................
\\
&&
(author sign)
\end{tabular}

\end{flushleft}
\newpage

\odsaditodzhora
\noindent {\bf Acknowledgement}

\todo[inline]{Podekovani}

\newpage
\tableofcontents

\newpage
\pagestyle{fancy}

\necislovana{Introduction} 
There are data all around us. As the society is becoming more and more digitalized the amount of the data is getting
bigger and bigger. A lot of enterprises, institutions and organizations realize that these data hide a huge potential
they can profit from. However the data themselves in their raw form are not usually sufficient to make a
conclusion from them. More often the data need to be processed and used as an inputs data for some kind of analyses. 
With the increasing number of gathered data a manual processing is almost inconceivable. Data are processed in an automatized way. 

Therefore, in order to be able to process the data independently of the type of acquisition, format or platform, 
it is necessary to define standards. Regarding spatial data, these standards are made by the Open Geospatial Consortium. 
Besides quite famous and used standards as WMS and WFS there also exists the WPS standard. The WPS standard defines an 
interface that facilitates the publishing of geospatial processes. It also provides rules how inputs and outputs are handled.
The WPS is only a standard and there are several implementations. This work is primarily focused on the \textit{PyWPS} framework.

\bigskip
The main topic of this thesis is process isolation. A process is just some geospatial operation which has its defined
inputs and outputs and which is deployed on a server. The server is able to execute multiple 
processes at the same time. This thesis deals with the isolation of individual processes especially for security and 
performance reasons. With every process fully isolated so they cannot interact with each other the higher security level
is assured.

The thesis is composed of several parts. The introductory research discusses the current state of the PyWPS and the
other projects that implement the WPS standard, namely \textit{ZOO-Project} and \textit{52$^{\circ}$North}. Then the
introducing research offers possible solutions to achieve process isolation. Various projects and technologies are
described and finally the Docker has been selected as the technology we try to implement in the practical part. Docker
has been selected as one of the most used technology for containerization. It puts every process into a separate
container so the isolation is ensured. Moreover Docker provides a mechanism to pause, stop and start a container so it
looks like a possible solution for the future WPS 2.0.0 standard implementation which requires this functionality. Using
Docker it also opens new possibilities, e. g. being able to deploy running job to cloud.

The technological background is covered in the second part. There is the WPS standard described, especially its operations
- \textit{GetCapabilities}, \textit{DesribeProcess} and \textit{Execute} - and inputs and outputs structures. There are also
\textit{PyPWS} and \textit{Docker} described.

Last part consists of the implementation description.
\todo[inline]{Doplnit uvod o implementaci}

I have chosen this topic to get in touch with another OGC standard. I also appreciate I can dive more into Docker technology
as it is a leader in containerization in the world.

\newpage
\part{Web Processing Service}
\newpage
\section{Web Processing Service}

\subsection{History}
The first mention of the Web Processing Service was in October 2004. Back then it
was named Geoprocessing Service \cite{OGC_news}. The specification was first 
implemented as a prototype in 2004 by Agriculture and Agri-Food Canada (AAFC).
In its further development during a Geoprocessing Services Interoperability Experiment \cite{WPS_experiment} 
the name was changed to "Web Processing Service" to avoid the acronym GPS, since 
this would have caused confusion with the conventional use of this acronym for 
Global Positioning System \cite{WPS_standart_1.0}. The first version of WPS was released in
September 2005 \cite{WPS_first}. The experiment demonstrated that various clients
could easily access and bind to services which were set up according to the WPS Implementation
specification.

Currently two major versions of WPS Standard exist. The WPS version 1.0.0 is currently used mostly.
If not explicitly said this thesis is dedicated to the version 1.0.0. The WPS version 2.0.0 was
released in 2015 \cite{WPS_second}.

\subsection{Open Geospatial Consortium}
The OGC \textit{Open Geospatial Consortium} is an international non-profit organization committed to making quality 
open standards for the global geospatial community. These standards are made through a consensus process and are freely available for anyone to use to improve sharing of the world's geospatial data. The OGC members come from government, commercial organizations, NGOs, academic and research organizations.\cite{OGC}

A predecessor organization, OGF, the Open GRASS Foundation, started in 1992. From 1994 the organization 
used the name \textit{Open GIS Consortium}, in 2004 the Board changed the name to \textit{Open Geospatial Consortium}.\cite{OGC_history}

Some of the widely-use OGC standards are:
\begin{itemize}
\item WCS, WMS, WFS, WMTS or WPS - standards for web services
\item GML, KML - standards for XML-based languages
\end{itemize}


\bigskip
\subsection{Web Processing Service}
The OGC Web -Processing Service (WPS) Interface Standard defines a standardized interface
that facilitates the publishing of geospatial processes. Also provides rules how to standardize
requests and responses for geospatial processing services. 

\textit{Process} means any operation on spatial
data from simple ones as maps overlay or buffering to highly complex as complicated global models. Any kind of GIS 
functionality can be offered to clients across a network with correctly configured WPS. 

\textit{Publishing} means
creating human-readable metadata that allow users to discover and use service as well as making 
available machine-readable binding information.

\textit{Data} can be both vector or raster data and can be delivered across the network or be available
at the server.

The interface does not specify any specific processes that can be implemented by a WPS nor any specific
data inputs or outputs. Instead it specifies generic mechanisms to describe any geospatial process and
data required and produced by the process. The interface does not only provide mechanisms for calculation
but also to identify required data, initiate the calculation and manage output data so clients can access it. 

\bigskip
Web Processing Service as one of the OGC web services specifies three types of requests which can be requested
by a client and performed by a WPS server. The implementation of these three requests is mandatory by all servers:

\textbf{\textit{GetCapabilities}} - The request returns to the client a Capabilities document that describes the abilities
of the specific server implementation. It also returns the name and abstract of each of the processes that can
be run on a WPS instance.

\textbf{\textit{DescribeProcess}} - The request returns details about the processes offered by a WPS instance. Describes
required inputs and produced outputs and their allowable formats.

\textbf{\textit{Execute}} - The request allows the client to run a specified process with provided parameters and returns
produced outputs.

\newpage
These operations are very similar to other OGC Web Services such as WMS, WFS, and WCS. Common interface aspects
are defined in the OpenGIS ® Web Services Common Implementation Specification \cite{OGC_common}. As seen in 
the class diagram at \\
Fig. \ref{fig:WPS_class_diagram} the WPS interface class inherits the GetCapabilities operation 
from OGCWebService interface class. The operations Execute and DescribeProcess are specific for the WPS. The WPS
operations are based on GET and POST requests.

\begin{figure}[h!]
\centering
\includegraphics[width=0.78\textwidth]{img/WPS_class_diagram.png}
\caption{WPS interface UML description, source: \cite{WPS_standart_1.0}}
\label{fig:WPS_class_diagram}
\end{figure}

The GetCapabilities and DescribeProcess shall use HTTP GET with KVP encoding and Execute operation shall use HTTP
POST with XML encoding. Summarized in Tab. \ref{tab:WPS_encoding}.
\begin{table}[h!]
\catcode`\-=12
\centering
\begin{tabular}{|c|c|c|}
\hline
\multirow{2}{*}{Operation} & \multicolumn{2}{c|}{Request encoding} \\ \cline{2-3} 
                           & Mandatory          & Optional         \\ \hhline{|=|=|=|}
GetCapabilities            & KVP                & XML              \\ \hline
DescribeProcess            & KVP                & XML              \\ \hline
Execute                    & XML                & KVP              \\ \hline
\end{tabular}
\caption{Operations request encoding}
\label{tab:WPS_encoding}
\end{table}

\subsubsection{GetCapabilities}
The GetCapabilities operation is mandatory. The operation allows a client to retrieve capabilities document (metadata)
from a server. The response XML document contains service metadata about the server and all implemented processes description.

\paragraph{GetCapabilities request}

\begin{table}[h!]
\catcode`\-=12
\centering
\begin{tabular}{|c|c|c|}
\hline
\thead{Name}               & \thead{Optionality and use} & \thead{Definition and format}    		\\ \hhline{|=|=|=|}
service=WPS                & Mandatory           & Service type identifier text 	\\ \hline
request=GetCapabilities    & Mandatory           & Operation name text              \\ \hline
AcceptVersion=1.0.0        & Optional            & Specification version            \\ \hline
Sections=All               & Optional            & \makecell{Comma-separated \\unordered list of sections} \\ \hline
updateSequence=XXX         & Optional            & \makecell{Service metadata \\document version}            \\ \hline
AcceptFormats=text/xml     & Optional            & \makecell{Comma-separated prioritized \\sequence of response formats} \\ \hline
\end{tabular}
\caption{GetCapabilities operation request URL parameters, source: \cite{OGC_common}}
\label{tab:WPS_GetCapabilities}
\end{table}

\begin{itemize}
\item\textit{service} - A mandatory parameter, WPS is only possible value.
\item\textit{request} - A mandatory parameter, GetCapabilities is only possible value.
\item\textit{version} - An optional parameter, version number. Three non-negative integers separated by a decimal point. Servers and
their clients should support at least one defined version.
\item\textit{sections} - An optional parameter that contains a list of section names. Possible values are: \textit{ServiceIdentification,
ServiceProvider, OperationsMetadata, Contents, All}.
\item\textit{updateSequence} - An optional parameter for maintaining the consistency of a client cache of the contents of a service
metadata document. The parameter value can be an integer, a timestamp, or any other number or string.
\item\textit{format} - An optional parameter that defines response format.
\end{itemize}

A client can request the GetCapabilities operation with parameters from the Tab. \ref{tab:WPS_GetCapabilities}. A corresponding
request URL looks like:

\noindent
\url{http://localhost:5000/wps?service=WPS&request=GetCapabilities&AcceptVe}\\
\url{rsion=1.0.0&Section=ServiceIdentification,OperationsMetadata&updateSeq}\\
\url{uence=XXX&AcceptFormats=text/xml}

\paragraph{GetCapabilities response}
\label{para:GetCapa_response}
\subparagraph{Normal response}
When GetCapabilities operation is requested a client retrieve service metadata document that contains sections specified in
\textit{sections} parameter. If the parameter value is \textit{All} or is not specified all sections retrieved.

\begin{itemize}
\item\textit{ServiceIdentification} - Server metadata.
\item\textit{ServiceProvider} - Server operating organization metadata.
\item\textit{OperationsMetadata} - Metadata about operations implemented by the WPS server, including URLs to request them.
\item\textit{ProcessOfferings} - List of processes with name and brief description implemented by the WPS server.
\end{itemize}

\bigskip
In addition to sections each GetCapabilities response should contains:
\begin{itemize}
\item\textit{version} - Specification version for GetCapabilities operation.
\item\textit{updateSequence} - Server metadata document version, value is increased whenever any change is made in complete service metadata document.
\end{itemize}

\subparagraph{GetCapabilities exceptions}
In case that WPS server encounters an error a client retrieve an exception report message with one of there exception code:

\begin{itemize}
\item\textit{MissingParameterValue} - GetCapabilities request does not contain a required parameter value.
\item\textit{InvalidParameterValue} - GetCapabilities request contains an invalid parameter value.
\item\textit{VersionNegotiation} - Any version from AcceptVersions parameter list does not match any version supported by the WPS server.
\item\textit{InvalidUpdateSequence} - Value of updateSequence parameter is greater than current value of service metadata updateSequence number.
\item\textit{NoApplicableCode} - Other exceptions.
\end{itemize}

\subsubsection{DescribeProcess}
The DescribeProcess operation is mandatory. The operation allows clients to retrieve a detailed description of one or more
processes implemented by a WPS server. The detailed information describes both required inputs and produced outputs and allowed
format.

\begin{table}[h!]
\catcode`\-=12
\centering
\begin{tabular}{|c|c|c|}
\hline
\thead{Name}               & \thead{Optionality} & \thead{Definition and format}    		\\ \hhline{|=|=|=|}
service=WPS                & Mandatory           & Service type identifier text 	\\ \hline
request=DescribeProcess    & Mandatory           & Operation name text              \\ \hline
version=1.0.0              & Mandatory           & WPS specification version            \\ \hline
Identifier=buffer          & Optional            & \makecell{List of one or more process\\ identifiers, separated by commas} \\ \hline
\end{tabular}
\caption{DescribeProcess operation request URL parameters, source: \cite{OGC_common}}
\label{tab:WPS_DescribeProcess}
\end{table}

\paragraph{DescribeProcess request}
\begin{itemize}
\item\textit{service} - Mandatory parameter, WPS is only possible value.
\item\textit{request} - Mandatory parameter, DescribeProcess is only possible value.
\item\textit{version} - Mandatory parameter, version number. Three non-negative integers separated by decimal point. Servers and
their clients should support at least one defined version.
\item\textit{Identifier} - Optional parameter, list of process names separated by comma. Another possible value is \textit{all}.
\end{itemize}

The DescribeProcess operation can be requested with parameters from Tab. \ref{tab:WPS_DescribeProcess}. A corresponding
request URL looks like: \url{http://localhost:5000/wps?request=DescribeProcess&service=WPS&identifier=all&version=1.0.0}

\paragraph{DescribeProcess response}
\label{para:DesribeProc_response}

\subparagraph{Normal response}
Normal response to DescribeProcess request contains one or more process descriptions for requested process identifiers in 
\textit{ProcessDescriptions} structure. Each
process description contains detailed information about process in \textit{ProcessDescription} (see Tab. \ref{tab:WPS_ProcessDescription})
including process inputs and outputs description. The number of inputs or outputs is not limited.

\bigskip
\begin{table}[h!]
\catcode`\-=12
\centering
\begin{tabular}{|c|c|c|}
\hline
\thead{Name}               & \thead{Optionality} & \thead{Definition and format}    		\\ \hhline{|=|=|=|}
Identifier      	       & Mandatory           & Process identigier             \\ \hline
Title 			           & Mandatory           & Process title 				  \\ \hline
Abstract		           & Optional            & Brief description              \\ \hline
Metadata		           & Optional            & \makecell{Reference to more metadata about this process} \\ \hline
Profile			           & Optional            & \makecell{Profile to which the WPS process complies} \\ \hline
processVersion	           & Mandatory           & Release version of process \\ \hline
WSDL    		           & Optional            & \makecell{Location of a WSDL document that \\describes this process} \\ \hline
DataInputs		           & Optional            & \makecell{List of the required and optional inputs} \\ \hline
ProcessOutputs	           & Mandatory           & \makecell{List of the required and optional outputs} \\ \hline
storeSupported	           & Optional            & \makecell{Complex data outputs can be \\stored by WPS server} \\ \hline
statusSupported	           & Optional            & \makecell{Execute response can be returned\\ quickly with status information} \\ \hline
\end{tabular}
\caption{Parts of ProcessDescription data structure, source: \cite{WPS_standart_1.0}}
\label{tab:WPS_ProcessDescription}
\end{table}

\newpage
\noindent
Three data types of input or outputs exist:
\begin{itemize}
\item\textit{LiteralData} - any string. It is used for passing single parameters like numbers or text parameters. There can be set
\textit{allowedValues} restriction. It can be a list of allowed values or input data type. Additional attributes such as \textit{units}
or \textit{encoding} can be set as well. 
\item\textit{ComplexData} - Complex data can be raster, vector or any file-based data, which are usually processed. ComplexData 
are often result of the process. The input can be specified more using mimeType, XML schema or encoding.
\item\textit{BoundingBoxData} - BoundingBox data are specified in OGC OWS Common specification as two pairs of coordinates (for 2D and 3D space). They can either be encoded in WGS84 or EPSG code can be passed too. They are intended to be used as definition of the target region.
\end{itemize}

\subparagraph{DescribeProcess exceptions}
In case that WPS server encounters an error a client retrieves an exception report message with one of there exception code:
\begin{itemize}
\item\textit{MissingParameterValue} - GetCapabilities request does not contain a required parameter value.
\item\textit{InvalidParameterValue} - GetCapabilities request contains an invalid parameter value.
\item\textit{NoApplicableCode} - Other exceptions.
\end{itemize}

\subsubsection{Execute}
The Execute operation is mandatory. The operation allows clients to run a specified process implemented by a server.
Inputs can be included directly in the request body or be referenced as a web-accessible resource. The outputs are returned
in XML response document, either directly embedded within the response document or stored as a resource accessible by
returned URL.

Usually the response document is returned right after the process execution is completed. However it is possible to get
response document right after sending a request. In this case, returned response document contains a URL link from which the
final response document can be retrieved after completed process execution. A client can request execution status update to
find out the amount of processing remaining if the execution is not completed. Shown in Fig. \ref{fig:WPS_sequence}.

\begin{figure}[h!]
\centering
\includegraphics[width=0.8\textwidth]{img/WPS_sequence.png}
\caption{Sequence diagram: a client requests storage of results, source: \cite{WPS_standart_1.0}}
\label{fig:WPS_sequence}
\end{figure}

\paragraph{Execute request}
Execute request is usually sent via HTTP POST request. Execute request contains parameters from Tab. \ref{tab:WPS_ExecuteRequest}. Example of Execute XML sent within the POST request can be found at App.
\ref{app:ExecuteRequest}. 

\begin{table}[h!]
\catcode`\-=12
\centering
\begin{tabular}{|c|c|c|}
\hline
\thead{Name}               & \thead{Optionality} & \thead{Definition and format}    		\\ \hhline{|=|=|=|}
service          	       & Mandatory           & Service type identifier text             \\ \hline
request			           & Mandatory           & Operation name text 				  \\ \hline
version			           & Mandatory           & WPS specification version              \\ \hline
Identifier   	           & Mandatory           & Process identifier \\ \hline
DataInputs		           & Optional            & \makecell{List of inputs provided \\ to this process execution} \\ \hline
ResponseForm	           & Optional            & Response type definition \\ \hline
language   		           & Optional            & Language identifier \\ \hline
\end{tabular}
\caption{Parts of Execute operation request, source: \cite{WPS_standart_1.0}}
\label{tab:WPS_ExecuteRequest}
\end{table}

\paragraph{Execute response}
Usualy the Execute operation response document is an XML document. The only exception is in case when a response form
of \textit{RawDataOutput} is requested, execution is successful and only one complex output is created, then directly
the produced complex output is returned.

In ussual case response to Execute opetation is an ExecuteResponse XML document. The contents depend on ResponseForm 
request elements.

\begin{table}[h!]
\catcode`\-=12
\centering
\begin{tabular}{|c|c|c|}
\hline
\thead{Name}               & \thead{Optionality} & \thead{Definition and format}    		\\ \hhline{|=|=|=|}
service          	       & Mandatory           & Service type identifier text             \\ \hline
version			           & Mandatory           & WPS specification version              \\ \hline
language   		           & Mandatory           & Language identifier \\ \hline
statusLocation	           & Optional            & \makecell{Reference to location where current\\ExecuteResponse document is stored} \\ \hline
serviceInstance	           & Mandatory           & \makecell{Reference to location where current\\ExecuteResponse document is stored} \\ \hline
Process			           & Mandatory           & Process description \\ \hline
Status			           & Mandatory           & Execution status of the process \\ \hline
DataInputs		           & Optional            & \makecell{List of inputs provided \\ to this process execution} \\ \hline
OutputDefinitions          & Optional            & \makecell{List of definitions of outputs \\desired from executing this process} \\ \hline
ProcessOutputs             & Optional            & \makecell{List of values of outputs \\ from process execution} \\ \hline
\end{tabular}
\caption{Parts of ExecuteResponse data structure, source: \cite{WPS_standart_1.0}}
\label{tab:WPS_ExecuteResponse}
\end{table}

\newpage
\section{WPS implementations}
The OGS WPS is just an interface standard that provides rules for standardizing requests and response. It also defines how clients can request the execution of defined processes and how the outputs are handled. There are several projects that implement this standard across the platforms or programming languages.

\subsection{deegree}
\textit{deegree} is open-source community-driven project for spatial data infrastructure written in Java. Besides from the other OGC Web Services it implements also WPS standard 1.0.0. Their solution offers sending request with KVP, XML or SOAP encoding, asyncronous/syncronous execution and API for implementing processes in Java. On their website there is a WPS demo
\footnote{\url{http://demo.deegree.org/wps-workspace/}} where all operations \textit{GetCapabilities}, \textit{DescribeProcess} and \textit{Execute} with various processes can be tested.

\bigskip
\begin{figure}[h!]
\centering
\includegraphics[width=0.2\textwidth]{img/deegree.png}
\caption{deegree project logo}
\label{fig:deegree}
\end{figure}

\subsection{52$^{\circ}$North WPS}
The \textit{52$^{\circ}$North} is the open-source software initiative. It is an international network of skilled specialists from research,
public administration or industry. They work on several projects and develop new technologies. Among their various specialization there is 
the 52$^{\circ}$North WPS project.

\bigskip
\begin{figure}[h!]
\centering
\includegraphics[width=0.2\textwidth]{img/Intro_52north.png}
\caption{52$^{\circ}$North project logo}
\label{fig:Intro_52north}
\end{figure}

The  WPS project is full Java-based open-source implementation of the WPS 1.0.0. The back-end side implements only version 1.0.0 and it
does not seem there is any progress in implementation of version 2.0.0. On the other hand on the 52$^{\circ}$North GitHub there is a
repository \textit{wps-js-client\footnote{\url{https://github.com/52North/wps-js-client}}} that is standalone Javascript WPS Client. 
The client enables building and sending requests against both WPS 1.0.0 and WPS 2.0.0 instances as well as reading the responses.

52$^{\circ}$North offers synchronous/asynchronous invocation with both HTTP-GET and HTTP-POST request. All results can be stored as
a web-accessible resource, WMS, WFS or WCS layer. Raw data inputs/outputs are also supported. Various extensions for different
computional backends exist: WPS4R (R Backend), GRASS extension, Sextante or ArcGIS Server Connector.

\subsection{GeoServer}
\textit{GeoServer} is Java-based server to store, view or edit geospatial data. Designed for interoperability, GeoServer conforms
all OGC standards. More famous WMS, WFS and WCS services are part of GeoServer core, however WPS implementation is available as 
extension.

\begin{figure}[h!]
\centering
\includegraphics[width=0.6\textwidth]{img/geoserver.jpg}
\caption{GeoServer logo}
\label{fig:geoserver_logo}
\end{figure}

The WPS extension is capable of direct reading and writing data from and to GeoServer. Therefore it is possible to create processes
based on inputs served from GeoServer as well as storing the outputs in the catalog.

Since GeoServer implements WPS standard 1.0.0, it must supports the \textit{GetCapabilities}, \textit{DescribeProcess} and \textit{Execute}
operations. Apart of these, it implements also \textit{GetExecutionStatus} and \textit{Dismiss} operations. The \textit{Dismiss} operation
serves for asyncronous requests to get progress report and eventually retrieve the result data. A client send in the GetExecutionStatus
request a mandatory \textit{executionId} parameter to specify the process. The executionId is mandatory parameter for Dismiss operation
either. The Dismiss operation cancels an execution of the process of given executionId. As seen in Fig. \ref{fig:geoserver_status},
GeoServer offers \textit{Progress status page} where progress of all executions can be reviewed as well as dismissing of each execution
can be done.

\begin{figure}[h!]
\centering
\includegraphics[width=\textwidth]{img/geoserver_status.png}
\caption{Process status page, source \cite{GS_docs}}
\label{fig:geoserver_status}
\end{figure}

\subsection{ZOO-Project}
\textit{ZOO-Project} is a WPS implementation writenn in C, Python and Javascript. It is an open-source project released under MIT licence.
The platform is composed of several components:

\begin{itemize}
\item WPS Server - ZOO-kernel is a server-side implementation written in C.
\item WPS Services - ZOO-services is a set of ready-to-use web services based on libraries such as \textit{GDAL}, \textit{GRASS GIS}
or \textit{CGAL}.
\item WPS API - ZOO-API is a server-side Javascript API for creating and chaining WPS web services.
\item WPS Client - ZOO-client is a client-side Javascript library for interacting with WPS Services.
\end{itemize}

ZOO-Project is the first and in this time probably only one full implementation of the WPS 2.0.0 standard. Apart from \textit{GetCapabilities},
\textit{DescribeProcess} and \textit{Execute} operations from WPS 1.0.0 standard it also implements \textit{GetStatus}, \textit{GetResult}
and \textit{Dismiss} operations from WPS 2.0.0.

To comply WPS 2.0.0 ZOO-Project must support synchronous/asynchronous invocation with both HTTP-GET and HTTP-POST request. To store 
results there is optional MapServer support. It is convenient to publish results directly as WMS, WFS or WCS resources.

\subsection{ArcGIS Server}
\textit{ArcGIS Server} is server GIS software developed by \textit{Esri}. It is capable of creating and managing GIS Web services, applications and data. It allows expose the analytic capability of ArcGIS to web as a \textit{Geoprocessing service}. A geoprocessing
service consist of one or more geoprocessing tasks which is ArcGIS geoprocessing tool running on the server. It is possible to publish Geoprocessing service with the WPS capabilities enabled, however only WPS 1.0.0 standard is supported.

\begin{figure}[h!]
\centering
\includegraphics[width=0.3\textwidth]{img/Intro_esri.png}
\caption{Esri logo}
\label{fig:geoserver_status}
\end{figure}

All published services have specified the minimum and maximum number of available instances. These instances run on the container machines within processes. The isolation level determines whether these instances run in separate processes or shared processes. 
\begin{itemize}
\item \textit{High isolation}- Fig. \ref{fig:arcgis_isolation_low} - each instance runs in its own process. If something causes the process to fail, it will only affect the single instance running in it.
\item \textit{Low isolation} - Fig. \ref{fig:arcgis_isolation_high} allows multiple instances of a service configuration to share a single process, thus allowing one process to handle multiple concurrent, independent requests. This is often referred to as multithreading.
\end{itemize} 

\begin{figure}[h!]
\centering
\begin{floatrow}
\ffigbox{\includegraphics[width=0.5\textwidth]{img/arcgis_isolation_low.png}}{\caption{Low isolation, source \cite{AG_docs}}}{\label{fig:arcgis_isolation_low}}
\ffigbox{\includegraphics[width=0.5\textwidth]{img/arcgis_isolation_high.png}}{\caption{High isolation, source \cite{AG_docs}}}{\label{fig:arcgis_isolation_high}}
\end{floatrow}
\end{figure}

The advantage of low isolation is that it increases the number of concurrent instances supported by a single process. Using low isolation can use slightly less memory on your server. However, this improvement comes with some risk. If a process experiences a shutdown or crash, all instances sharing the process are destroyed. It is strongly recommended that you use high isolation.\cite{AG_docs}

\subsection{PyWPS}
PyWPS is a server-side implementation of the WPS standard written in Python. This project will be described in depth in the Sec. \ref{sec:PyWPS}

\newpage
\part{PyWPS}

\newpage
\section{PyWPS}
\label{sec:PyWPS}
\subsection{History}
The origin of PyWPS started in 2006 as a student project. The first presentation was held at the FOSS4G 2006 conference in Lausanne titled 
‘GRASS goes to web: PyWPS’. During November 2006 the version 1.0.0 was released together with WUIW and Embrio projects that brought the
funcionality of GRASS GIS and general web interface able to handle any WPS server.\cite{PyWPS_presentation}\cite{PyWPS_web}

In 2007 PyWPS 2.0.0 was released supporting WPS standard 0.4.0. New version improved stability a approached on the standard implementation. It came with new WPS client and WPS plugin for OpenLayers \footnote{\url{http://openlayers.org/}}.

Next year in 2008 PyWPS 3.0.0 was released with support for WPS 1.0.0. It was possible to run multiple WPS instances
with one PyWPS installation. This version had simple code structure and contained examples of processes. 

The newest version is PyWPS 4.0.0 from 2016 when PyWPS-4 branch was merged to official PyWPS repository as its master branch.
 New version is described in following Sec. \ref{sec:PyWPS4}.

\begin{figure}[h!]
\centering
\includegraphics[width=0.4\textwidth]{img/pywps_logo.png}
\caption{PyWPS project logo}
\label{fig:pywps_logo}
\end{figure}

\subsection{PyWPS 4.0}
\label{sec:PyWPS4}
PyWPS-4 is the most current version of PyWPS. Rewriting from scratch involved this major changes:
\begin{itemize}
\item It is written in \textit{Python 3} with backward support for Python 2.7.
\item It utilizes native Python bindings to existing projects (GRASS GIS).
\item New popular formats like \textit{GeoJSON}, \textit{KML} or \textit{TopoJSON} are reflected and their support is provided.
\item PyWPS project has changed the license from \textit{GNU/GPL} to \textit{MIT}.
\item PyWPS 4.0 is implemented using the \textit{Flask} framework.
\item A C-based XML parser \textit{Lxml} is used to handle XML files.
\item \textit{OWSLib} structures are used for some data types.
\end{itemize}

\subsection{PyWPS-demo}
PyWPS-demo is a small side project distributed with PyWPS. It is a simple demo instance of PyWPS server running on 
\textit{Flask}\footnote{\url{http://flask.pocoo.org/}}. Flask is a microframework for web applications in Python. 
Flask provides built-in development server and debugger and RESTful request dispatching. Starting PyWPS-demo server with Flask
is very simple and can be done with command in Lst. \ref{lst:demo_start}. After starting the PyWPS-demo server the PyWPS homepage can be 
visited at: \url{http://localhost:5000}

\bigskip
\begin{lstlisting}[basicstyle=\small,caption={Starting PyWPS-demo server},label={lst:demo_start}]
python3 demo.py
\end{lstlisting}

PyWPS-demo comes with several demo processes:
\begin{itemize}
\item \textit{area.py} - Process calculates area of given polygon.
\item \textit{bboxinout.py} - Process transforms bounding box to another EPSG.
\item \textit{buffer.py} - Process returns buffers around the input features, using the GDAL library.
\item \textit{centroids.py} - Process returns a GeoJSON with centroids of features from  an uploaded GML.
\item \textit{feature\_count.py} - Process counts the number of features in an uploaded GML.
\item \textit{grassbuffer.py} - Process uses the  GRASS GIS \textit{v.buffer} module to generate buffers around inputs.
\item \textit{sayhello.py} - Process returns a literal string output with Hello plus the inputed name.
\item \textit{sleep.py} - Process will sleep for a given delay or 10 seconds if not a valid value.
\item \textit{ultimate\_question.py} - The process gives the answer to the ultimate question of "What is the meaning of life?"
\end{itemize}

Except these example processes the demo offers also example configuration file. Configuration file contains several parameters in
these four sections:
\begin{itemize}
\item \textit{metadata} - parameters containing information for metadata creation.
\item \textit{server} - definition of path to workdir and output directories, maximum number of parallel running or stored processes. 
\item \textit{logging} - logging level setting, path to log file and log database.
\item \textit{grass} - GRASS setting.
\end{itemize}



\newpage
\section{Process isolation in PyWPS}
\subsection{Asynchonous requests}
Right now in PyWPS 4.0 version a PyWPS server instance is able to run multiple
concurrent processes in parallel. The server is configured for maximal amounts of concurrently running processes at
the same time and for the maximal amount of waiting processes in a queue, to later start their execution once new
slots are available. If the new Execute request is received and the maximal amount is exceeded, the request is rejected
and user is informed in response (see Lst. \ref{lst:Isolation_rejected}).

\begin{lstlisting}[basicstyle=\small,caption={Resource exceeded exception},label={lst:Isolation_rejected}]
<?xml version="1.0" encoding="UTF-8"?>
<ows:ExceptionReport xmlns:ows="http://www.opengis.net/ows/1.1" version="1.0.0">
    <ows:Exception exceptionCode="ServerBusy">
        <ows:ExceptionText>
            Maximum number of parallel running processes reached. Please try later.
        </ows:ExceptionText>
    </ows:Exception>
</ows:ExceptionReport>
\end{lstlisting}

To facilitate the management of concurrent processes, process metadata are stored into a local database. This database is used
for logging and saving waiting Execute requests in the queue and invoking them later on.
This database will also enable the implementation of pausing, releasing and deleting running process. These features will
allow PyWPS to comply with WPS version 2.0.0.

\subsection{Current state}
\label{subsec:current_state}
At the beginning of every process execution its own temporary directory \textit{workdir} is created. During the execution
temporary files and continuous outputs are stored in this folder. After successful execution final outputs are
moved to \textit{outputs} directory. Both directories \textit{outputs} and \textit{workdir} are configurable and user can
change path to them.

\bigskip
\begin{lstlisting}[basicstyle=\small,caption={pywps.cfg - mode parameter}]
[processing]
mode=multiprocessing
\end{lstlisting}

Current version of PyWPS offers two solutions for running parallel processes:
\begin{itemize}
\item Multiprocessing
\item Job Scheduler Extension\footnote{Job Scheduler Extension is currently only in develop branch of PyWPS.}
\end{itemize}

If the execute request is sent asynchronously the type of process constructor is chosen depending on configuration
parameter \textit{mode} in section \textit{processing} which is by default \textit{multiprocessing} or can be changed 
to \textit{scheduler}.

\bigskip
\begin{lstlisting}[basicstyle=\small,caption={processing.\_\_init\_\_.py},label={lst:Process_init},language=python]
def Process(process, wps_request, wps_response):
    """
    Factory method (looking like a class) to return the
    configured processing class.

    :return: instance of :class:`pywps.processing.Processing`
    """
    mode = config.get_config_value("processing", "mode")
    LOGGER.info("Processing mode: %s", mode)
    if mode == SCHEDULER:
        process = Scheduler(process, wps_request, wps_response)
    else:
        process = MultiProcessing(process, wps_request, wps_response)
    return process
\end{lstlisting}
\bigskip
\bigskip

\paragraph{Multiprocessing}
By default for  processes   running   in   the   background,   the   Python \textit{multiprocessing} module is used – 
this makes it possible to use PyWPS on the Windows operating system too.

\begin{figure}[h!]
\centering
\includegraphics[width=0.7\textwidth]{img/Diag_multiprocessing.png}
\caption{Sequence diagram: Multiprocessing}
\label{fig:Diag_multiprocessing}
\end{figure}

The number of processes running in parallel is configirable by parameter \textit{parallelprocesses} of section \textit{server} in 
configuration file. In the Fig. \ref{fig:Diag_multiprocessing} are shown two running processes. A client sends an Execute request to
a server. Server sends back to the client an ExecuteResponse that \textit{Process1} (green in the figure) was accepted and starts its 
execution. During the execution the process updates its status. The interval of status updates depends on the code of the Process1.
Process1 must support status update otherwise it cannot be run in asyncronous mode.

During the execution of Process1 server receives another Execution request. It sends back the Execution response and starts the execution
of \textit{Process2} (blue in the figure). Separated Python \textit{Process}\footnote{Explanation of term Python \textit{Process} and its
differences to \textit{Thread} is in next paragraph.} is created. Both of the processes run on the host machine but both have own memory
space. Their executions run concurently and client can request their status. In the figure, the Process2 ended first and client can retrieve
the result from the server. Once the Process1 ends too, the client can retrieve its result from the server as well.

It is important to say that in case of multiprocessing processes run concurently with its own memory space nevertheless they are not isolated.
They run on the same host machine and share the resources. There are even methods like \textit{Pipe()} that enables communication between 
processes.

\paragraph{Process vs Thread} In Python there are two ways to achieve \textit{pararellism}. It is \textit{multiprocessing}
\footnote{\url{https://docs.python.org/3/library/multiprocessing.html}} with using processes and \textit{threading}\footnote{https://docs.python.org/3/library/threading.html} with threads. The main difference is that threads run in the same memory space, while processes
have separate memory. Multiprocessing takes advantages of multiple CPUs and cores while threads are more lightweighted and have low memory
footprint. In case of PyWPS asyncronous requests, for every execution its own process with its own memory space is created.

\paragraph{Job Scheduler Extension}
PyWPS scheduler extension offers possibilities to execute asynchronous processes out of the WPS server machine.
This extension enables to delegate execution of processes to a scheduler system like \textit{Slurm}, \textit{Grid Engine} 
and \textit{TORQUE} from Adaptive Computing. These schedular systems are usually located at \textit{High Performance Compute (HPC)}
centers.

\begin{figure}[h!]
\centering
\begin{floatrow}
\ffigbox{\includegraphics[width=0.33\textwidth]{img/Isolation_grid.jpg}}{\caption{Grid Engine}}{\label{fig:Isolation_grid}}
\ffigbox{\includegraphics[width=0.25\textwidth]{img/Isolation_slurm.png}}{\caption{Slurm}}{\label{fig:Isolation_slurm}}
\end{floatrow}
\end{figure}

\begin{figure}[h!]
\centering
\includegraphics[width=0.33\textwidth]{img/Isolation_torque.png}
\caption{TORQUE}
\label{fig:Isolation_torque}
\end{figure}
\bigskip

The PyWPS scheduler extension uses the Python \textit{dill} library to dump and load the processing job to/from filesystem. The batch script executed on the scheduler system calls the PyWPS \textit{joblauncher} script with the dumped job status and executes the job (no WPS service running on scheduler). The job status is updated on the filesystem. Both the PyWPS service and the joblauncher script use the same PyWPS configuration. The scheduler assumes that the PyWPS server has a shared filesystem with the scheduler system so that XML status documents and WPS outputs can be found at the same file location. The interaction diagram how the communication between PyWPS and the scheduler works is displayed in Fig. \ref{fig:Isolation_interaction}.

\bigskip
\begin{figure}[h!]
\centering
\includegraphics[width=\textwidth]{img/Isolation_slurm_usage.png}
\caption{Example of PyWPS scheduler extension usage with Slurm, source: \cite{PyWPS_docs}}
\label{fig:Isolation_slurm_usage}
\end{figure}

\begin{figure}[h!]
\centering
\includegraphics[width=\textwidth]{img/Isolation_interaction.png}
\caption{Communication between PyWPS and scheduler, source: \cite{PyWPS_docs}}
\label{fig:Isolation_interaction}
\end{figure}

\subsection{Possible solutions}
In previous section there were described two mechanisms for running parallel processes. Nevertheless in case of Python module
\textit{Multiprocessing} the processes are not really isolated. They run concurrently but they can share resources and there are 
even methods like \textit{Pipe()} that enables communication between processes.

On the otherhand \textit{Job Scheduler Extension} is dependent on \textit{dill} library as well as on some external scheduler
systems like \textit{Slurm}, \textit{Grid Engine} or \textit{TORQUE}.

\bigskip
In this section there are described some other solutions. Some of them were suggested by PyPWS developers with encouragement to
make a feasible study. Some of them were discovered during research on the internet forums like StackOverflow, some of them
were referenced in the documentation of other projects. During the research two requirements were considered.

\begin{itemize}
\item The solution provides a mechanism for full isolation. This is a must-have requirement.
\item The solution provides a mechanism for start/pause/stop process execution. This is a nice-to-have requirement as this functionality
will be required to comply WPS 2.0.0 standard.
\end{itemize}

\noindent
Finnaly these solutions were considered:
\begin{itemize}
\item Celery
\item Docker
\item psutil
\item SandboxedPython
\item VM
\end{itemize}

\subsubsection{Celery}
\textit{Celery} is a task queue system written in Python. It helps distribute work across threads and even machines. Basic term is
a \textit{task}. A task is a unit of work and it is an input into the task queue. The task queue is constantly monitored for new 
work to perform.

To communicate between client and workers Celery uses a \textit{broker}. The communication is via messages. To initiate a task the
client adds a message to the queue and the broker then delivers the message to a worker. Multiple workers and brokers can be added
so there is assured high availability and horizontal scaling.

Celery provide worker remote control client in class \textit{celery.app.control.Control(app=None)}. The class offers these functions:
\begin{itemize}
\item\textbf{revoke} - Tell all (or specific) workers to revoke a task by id. If a task is revoked, the workers will ignore the task and not execute it after all.
\item\textbf{shutdown} - Shutdown worker(s).
\item\textbf{terminate} - Tell all (or specific) workers to terminate a task by id.
\end{itemize}

\subsubsection{Docker}
\textit{Docker} is one of the most used technology regarding containerization. This technology is described in depth in a later chapter.

\subsubsection{psutil}
\textit{psutil} is Python library for process and system management. It handles system monitoring, limiting process resources
and the management of running processes. Its implementation is based on UNIX command line tools. psutil offers functions applied
to these sections:
\begin{itemize}
\item CPU - functions for CPU statistics such as CPU utilization percentage, frequency and others. 
\item Memory - functions for system memory usage and swap memory statistics.
\item Disks - functions for disk statistics such as disk usage or disk IO operations counter.
\item Network - functions for network IO operations or network connection statistics.
\item Sensors - functions for statistics about fans, battery or hardware temperature.
\item Others - functions for boot time and users statistics.
\item Processes - functions will be described in detail later.
\end{itemize}

\paragraph{Processes} - Class \textit{psutil.Process(pid=None)} represents an OS process with given pid. The class is
bound with a process via its PID. The \textit{Process} class offers these methods for starting/pausing:
\begin{itemize}
\item suspend() - The method suspends a process using SIGSTOP signal.
\item resume() - The method resumes a process using SIGCONT signal.
\item terminate() - The method terminates a process using SIGTERM signal.
\item kill() - The method kills a process using SIGKILL signal.
\end{itemize}

\subsubsection{Sandboxed Python}
The general goal of a sandbox is to run applications securely inside isolated environment they cannot escape from and
affect other parts of the system. Developers use them to run untrusted code inside. It is quite difficult to develop
fully sandboxed solution due to Python complexity. The basic problem is that Python introspection allows several ways
to escape out of the sandbox. True security requires an overall design with many security considerations included. Some
of the projects that can run Python code in a sandbox are:
\begin{itemize}
\item PyPy
\item Jython
\end{itemize}

\paragraph{PyPy} PyPy is Python interpreter written in RPython that implements full Python language and very closely 
emulates the behavior of CPython. PyPy offers fully portable sandboxing feature similar to OS-level sandboxing (e. g. SECCOMP).
It is not sandboxing at the Python language level so it does not put any restriction on any Python functionality.

Untrusted Python code that is intended to be sandboxed is launched in a subprocess, that is a special sandboxed version of PyPy.
All its inputs/outputs are not directly performed but are serialized to a stdin/stdout pipe. The outer process reads the pipe 
and afterward decides which commands are allowed.

\paragraph{Jython} Jython is Python language interpreter for Java. Java offers strong sandboxing mechanisms. The security 
facility in Java that supports sandboxing is the \textit{java.lang.SecurityManager}. By default, Java runs without a 
SecurityManager.

\paragraph{pysandbox} A prove, that it is very difficult to develop some kind of sandbox with all security holes considered, could be a project  \textit{pysandbox}\footnote{\url{https://github.com/vstinner/pysandbox}}. After working on it for 3 years, during which
the project was used on various production servers by other developers, its author declared that the project is broken by
design. In his post to the python-dev mailing list \cite{PyDev_ML} the author explained that with every vulnerability founded
it became more difficult to actually write a real code:

\textit{
"To protect the untrusted namespace, pysandbox installs a lot of
different protections. Because of all these protections, it becomes
hard to write Python code. Basic features like "del dict[key]" are
denied. Passing an object to a sandbox is not possible to sandbox,
pysandbox is unable to proxify arbitary objects.}

\textit{
For something more complex than evaluating "1+(2*3)", pysandbox cannot
be used in practice, because of all these protections. Individual
protections cannot be disabled, all protections are required to get a
secure sandbox."}

\subsubsection{Virtual Machine/Vagrant}
Using full virtualization for process isolation is mentioned here but in fact it is hard to imagine this solution could work
in practice. \textit{Vagrant} is a tool for managing and building virtual machines. It provides a way how to manage various virtual 
machines in an automatized way e. g. using scripts. There also exists a Python package \textit{python-vagrant} that offers 
Python bindings for interacting with Vagrant.

However in our use-case using Vagrant would mean that for every process execution a separate virtual machine is created. Depending
on the process algorithm complexity the process execution can last from milliseconds to hours or days. On the other hand building 
a virtual machine and booting into it last at least few seconds. That is why it is hard to imagine using virtual machine, which takes
few seconds to boot up, to isolate process, which execution lasts less than a second.

\newpage
\section{Docker}
\textbf{Containerization} is a lightweight alternative to full machine virtualization. It involves encapsulating an application 
into a container with its own operating environment. It helps to run a containerized application on any physical machine without any
worries about dependencies. The origin of containerization lies in the \textit{LinuX Containers {LXC}} format. Containerization
works only in Linux environments and can run only Linux applications.

\begin{figure}[h!]
\centering
\includegraphics[width=0.33\textwidth]{img/Docker_logo}
\caption{Docker logo}
\label{fig:Docker_logo}
\end{figure}

Docker is not the only one technology for containerization. Other alternatives exist, it is \textit{Kubernets}, \textit{CoreOS rkt}, 
\textit{Open Container Initiative (OCI)}, \textit{Canonical's LXD}, \textit{Apache Mesos and Mesosphere} and many others. 
However Docker is a leader on the field of containerization and with most public traction is de facto considered as a container standard.
That's why the Docker was chosen for this thesis as a container technology. So from this point on any term \textit{container} refers to
Docker container.

\begin{figure}[h!]
\centering
\begin{floatrow}
\ffigbox{\includegraphics[width=0.2\textwidth]{img/Docker_kubernetes.png}}{\caption{Kubernetes}\label{fig:Docker_kubernetes}}
\ffigbox{\includegraphics[width=0.2\textwidth]{img/Docker_rkt.png}}{\caption{CoreOS rkt}\label{fig:Docker_rkt}}
\end{floatrow}
\end{figure}

\begin{figure}[h!]
\centering
\begin{floatrow}
\ffigbox{\includegraphics[width=0.13\textwidth]{img/Docker_lxd.png}}{\caption{Canonical's LXD}\label{fig:Docker_lxd}}
\ffigbox{\includegraphics[width=0.25\textwidth]{img/Docker_mesos.jpg}}{\caption{Apache mesos}\label{fig:Docker_mesos}}
\end{floatrow}
\end{figure}

\paragraph{Docker} is a Linux container technology that allows package and ship applications and everything it needs to execute into a standard
format, and run them on any infrastructure.

\subsection{Virtual machine vs. Docker container}
Both virtual machines and Docker containers are two ways how to deploy multiple, isolated applications on a single platform. They
both offer a way to isolate an application and its dependencies into a self-contained unit that can run anywhere. They both offer
some kind of virtualization. They differ in architecture, see Fig. \ref{fig:Docker_VM}, \ref{fig:Docker_container}.

\begin{figure}[h!]
\centering
\begin{floatrow}
\ffigbox{\includegraphics[width=0.495\textwidth]{img/Docker_VM.png}}{\caption{Virtual machine architecture, source \cite{Docker_docs}}\label{fig:Docker_VM}}
\ffigbox{\includegraphics[width=0.5\textwidth]{img/Docker_container.png}}{\caption{Containers architecture, source \cite{Docker_docs}}\label{fig:Docker_container}}
\end{floatrow}
\end{figure}

\subsubsection{Virtual machine}
Let's start with a virtual machine (Fig. \ref{fig:Docker_VM}) and its layers description from the bottom up:
\begin{itemize}
\item \textit{Infrastructure} - It can be a PC, developer's laptop, a physical server in datacenter but as well a virtual private
server in the cloud as Microsoft Azure or Amazon EC2.
\item \textit{Host OS} - Host operating system. In case of native hypervisor this layer is missing. In case of hosted hypervisor
it is probably some distribution of Linux, Windows or MacOS.
\item \textit{Hypervisor} - Also called virtual machine monitor (VMM). It allows hosting several different virtual machines
on a single hardware. There are two types of hypervisors:
\begin{itemize}
\item Type 1 -  Also called \textit{bare metal} or \textit{native}. This type is run on the host's hardware to control it as well as manage 
the virtual machines on it. It is much faster and more efficient. This type hypervisors are KVM, Hyper-V or HyperKit.
\item Type 2 - So called \textit{embedded} or \textit{hosted} hypervisors. These hypervisors are run on a host OS as a software. They are slower
and less efficient on the other hand they are much easier to set up. It includes VirtualBox or VMWare Workstation.
\end{itemize}
\item \textit{Guest OS} - Guest operating system. Each VM requires own guest operating system which is controlled by the hypervisor. Each 
guest OS needs its own CPU and memory resources and starts on hundreds of megabytes in size.
\item \textit{Bins/Libs} - Each guest OS needs various binaries and libraries for running the application. It can be \textit{python-dev} or \textit{default-jdk} packages as well as personal packages to run the application.
\item \textit{Application} - The application source code that is desired to be run isolated. Therefore each application or each version of the application has to be run inside of its own guest OS with own copy of bins and libs. 
\end{itemize}

\subsubsection{Docker container}
\noindent
Now, what is different regarding containers (Fig. \ref{fig:Docker_container})
\begin{itemize}
\item \textit{Infrastructure} - PC, laptop, physical or virtual server.
\item \textit{Host OS with container support} - Any OS capable of run Docker. All major distributions of Linux are supported and there are ways
to run Docker even on MacOs and Windows too.
\item \textit{Docker engine} - Also called Docker daemon. It is a service that runs in the background on host operating system. It manages all interaction with containers.
\item \textit{Bins/Libs} - Binaries and libraries required by the application. They get built into special packages called \textit{Docker images}.
The Docker daemon runs those images.
\item \textit{Application} - Each application and its library dependencies get packed into the same Docker image. It is managed independently by the Docker daemon. 
\end{itemize}

\noindent
But the architecture is not the only one difference:
\begin{itemize}
\item Docker uses Docker daemon to manage containers, hypervisor manages virtual machines.
\item The Docker daemon communicates directly with host OS and manage resources for each container.
\item VMs usually boot up in a minute and more, containers start in seconds.
\item Docker virtualizes operating systems, using VMs is hardware virtualization.
\item VM and container vary in size. VMs start at hundreds of megabytes. A container can be smaller than one megabyte.
\item Containers share the kernel although they are isolated. VMs are monolithic and stand-alone.
\end{itemize}

\subsection{Dockerfile}
Dockerfile is a core file that contains the instruction to be performed when an image is built. It usually consists of commands to install packages, calls to other scripts, setting environmental variables, adding files or setting permissions. In Dockerfile there is also defined what image is to be used as a base image for the build.

\paragraph{Dockerfile instructions}
\begin{itemize}
\item \textit{FROM} - The FROM instruction defines the base image for next instructions and initializes a new build stage. Every Dockerfile
has to start with FROM command. The only exception is ARG command which can be before FROM command.
\item \textit{ARG} - The ARG instruction defines a variable that users can pass at build-time to the builder.
\item \textit{ENV <key>=<value>} - The ENV instruction sets the environment variables. It is key-pair value. 
\item \textit{LABEL} - The LABEL instruction adds metadata to an image. A LABEL is a key-value pair. It can be anything from version number to a description.
\item \textit{ADD <src> <dest>} - The ADD instruction copies files or directories from source and adds them at the destination path. It also unzips or untars files when added.
\item \textit{COPY <src> <dest>} - Similar to the ADD instruction it copies files or directories from source and adds them to the destination path. This command doesn't provide any kind of decompression.
\item \textit{RUN <command>} - The RUN instruction will execute any defined command and commit the results.
\item \textit{CMD ["executable","param1","param2"]} - The CMD instruction provides defaults for an executing container. It can include an
executable. In case the executable is omitted the CMD instruction must be used together with the ENTRYPOINT instruction. There can be only
one CMD instruction in Dockerfile. In case there is more CMD the last one will be used.
\item \textit{ENTRYPOINT} - The ENTRYPOINT defines a container configuration that will run as executable.
\item \textit{WORKDIR /path/to/dir} - The WORKDIR instruction sets the working directory for any RUN, CMD, COPY and ADD instruction that follows in Dockerfile.
\item \textit{EXPOSE} - The EXPOSE instruction informs Docker that the container listens on the specified network ports at runtime.
\item \textit{VOLUME} - The VOLUME instruction creates a mount point with the specified name and marks it as holding externally mounted volumes from the native host or other containers.
\end{itemize}

Except for the FROM instruction, all the instructions can be defined from the command line when starting docker container. There are more Dockerfile instructions however they are not relevant to this thesis as there are never used in practical part.

\newpage
\part{Implementation}
\newpage
\section{Operations overview}
\label{sec:operations_ov}
PyWPS in current version 4.0.0 implements all mandatory operations: \textit{Execute}, \textit{GetCapabilities}, \textit{DescribeProcess}.
Operations are handled by corresponding methods \textit{execute()}, \textit{get\_capabilities()} and \textit{describe()} in \textit{Service.py} class. 

However both \textit{GetCapabilities} and \textit{DescribeProcess} operations run in syncronous mode only. After sending a request,
a client receives back GetCapabilities or DescribeProcess response (both detaily described in \ref{para:GetCapa_response} and \ref{para:DesribeProc_response}). Both operations return only information or description about process but do not trigger the
execution of the process. It is supposed the response to GetCapabilities and DescribeProcess is returned almost immediately.
During the GetCapabilities and the DescribeProcess operations a process execution is not started and therefore there is no
starting process to be isolated. That is why from this point on this thesis is dedicated only to \textit{Execute} operation.

\begin{figure}[h!]
\centering
\includegraphics[width=0.9\textwidth]{img/Diag_operations.png}
\caption{PyWPS operations activity diagram}
\label{fig:Diag_operations}
\end{figure}

\section{Execute operation}
\subsection{Service.execute()}
As mentioned in previous section Sect. \ref{sec:operations_ov}, \textit{Execute} operation is handled by \textit{execute()} method.
Inputs for the method are:
\begin{itemize}
\item \textit{identifier} (string) - a name of the process which execution is requested and which is supported by WPS server.
\item \textit{wps\_request} (WPSRequest object) - an object containing original HTTP request.
\item \textit{uuid} (integer) - unique identifier of process execution.
\end{itemize}

\begin{figure}[h!]
\centering
\begin{floatrow}
\ffigbox{\includegraphics[width=0.4\textwidth]{img/Diag_service_execute.png}}{\caption{Activity diagram: method \textit{Service.execute()}}\label{fig:DiagServiceExecute}}
\ffigbox{\includegraphics[width=0.4\textwidth]{img/Diag_service_parse_execute.png}}{\caption{Activity diagram: method \textit{Service.\_parse\_and\_execute()}}\label{fig:DiagServiceParseExecute}}
\end{floatrow}
\end{figure}

The flowchart of the process execution is displayed at Fig. \ref{fig:DiagServiceExecute}. 
At first a deepcopy of the process instance is created so that procesesses cannot override each other. Then a temporary working directory \textit{workdir} is created and set as a current workdir for the process execution. To the workdir all input files are
copied as well as all temporary files and outputs are stored here. Then the method \textit{\_parse\_and\_execute()} is called (see
Fig. \ref{fig:DiagServiceParseExecute}). Here the inputs are parsed, in case of web-referenced input, the data are downloaded to workdir, in case of directly in request sent data, the data are saved into a file in workdir. The process execution afterward runs in \textit{Process.execute()} method. This method returns a \textit{wps\_response} - an instance of \textit{WPSReponse} object.

\subsection{Process.execute()}
The method \textit{execute()} of class \textit{Process} contains crucial if-statement where is decided whether the process will be
run in asyncronous or syncronous mode. Running in asyncronous mode can be enforced by setting both attributes \textit{status} and \textit{storeExecuteResponse} of the \textit{ResponseDocument} element in the ExecuteRequest XML to True.

\bigskip
\begin{lstlisting}[basicstyle=\small,caption={ReponseForm element of ExecuteRequest XML},language=XML,label={lst:Execute_ResponseForm}]
<wps:ResponseForm>
 <wps:ResponseDocument status="true" storeExecuteResponse="true">
  <wps:Output asReference="true">
   <ows:Identifier>buff_out</ows:Identifier>
  </wps:Output>
 </wps:ResponseDocument>
</wps:ResponseForm>
\end{lstlisting}
\bigskip 

No matter whether the process runs syncronously or asyncronously always there is a control how many parallel processes are currently
running. The number of the maximum of concurrently running processes can be configured. If the process is asyncronous and the number of currently running processes exceeds the maximal number, the process is stored and its execution is started lately. In case of the syncronous process the \textit{ServerBusy} exception is raised. If the number of processes is smaller than the maximal number of 
concurrent processes, the process can be executed. In syncronous mode the \textit{\_run\_process()} is called, in asyncronous mode the method \textit{\_run\_async()} is called. The activity diagram of the \textit{Process.execute()} is displayed in Fig.\ref{fig:Diag_process_execute}.

\begin{figure}[h!]
\centering
\includegraphics[width=0.9\textwidth]{img/Diag_process_execute.png}
\caption{\textit{Activity diagram: Process.execute()}}
\label{fig:Diag_process_execute}
\end{figure}

\subsection{Processing module}
Until now the code described in this thesis was not modified. Requirements which have been considered during the implementation of
Docker technology were that the source code will not be very modified, the process isolation will be easily inserted and the project structure will be kept the same. Keeping this in mind changes in source code were made only in \textit{processing} module.

As mentioned in Sec. \ref{subsec:current_state}, PyWPS uses solely the Python package \textit{Multiprocessing} in production version.
In develop branch there is also \textit{Scheduler} extension as one of the option for multiprocessing. In this thesis another option 
\textit{Docker} for processing was added. The desired option for processing can be configured in configuration file via parameter \textit{mode} in section \textit{processing} (see Lst. \ref{lst:Mode_config}), possible values are:
\begin{itemize}
\item docker
\item scheduler
\item multiprocessing - default option
\end{itemize}

\bigskip
\begin{lstlisting}[basicstyle=\small,caption={Processing mode configuration},language=XML,label={lst:Mode_config}]
[processing]
mode=docker/scheduler/multiprocessing
\end{lstlisting}

\begin{figure}[h!]
\centering
\begin{floatrow}
\ffigbox{\includegraphics[width=0.48\textwidth]{img/Diag_run_async.png}}{\caption{Activity diagram: Method \textit{Process.\_run\_async() }}}{\label{fig:Diag_run_async}}
\ffigbox{\includegraphics[width=0.48\textwidth]{img/Diag_class_processing.png}}{\caption{Class diagram: \textit{Processing} class}}{\label{fig:Diag_class_proc}}
\end{floatrow}
\end{figure}

\bigskip
The whole Docker implementation is in \textit{Container.py}. The class \textit{Container} handles containers creation, interaction with
server, file-system mounting and all container management.

\newpage
\section{\textit{Container} class}
The main idea of process isolation using Docker is quite simple. For every process execution one separate Docker container is created.
Instead of starting process execution on the host PyWPS server after receiving ExecuteRequest from the client, the ExecuteRequest is
forwarded to PyWPS server running inside Docker container. The process execution runs inside the container. After successful process
execution the outputs are available at the host server. The host server and the container share the same process workdir at filesystem.

\bigskip
\begin{figure}[h!]
\centering
\includegraphics[width=0.9\textwidth]{img/Diag_sequence.png}
\caption{Sequence diagram: Process execution using Docker}
\label{fig:Diag_sequence}
\end{figure}

\subsection{\textit{Container} class constructor}
\label{sub:Container_init}
\textit{Container} class is initialized with standard Python method \textit{\_\_init\_\_()}. As an inheritor of \textit{Processing}
class, at first the parent constructor is called with \textit{super().\_\_init\_\_()} method. Follows description of methods which are
called inside the constructor method.

\bigskip
\begin{lstlisting}[basicstyle=\small,caption={\textit{Container} class constructor},label={lst:Container_constructor}]
def __init__(self, process, wps_request, wps_response):
    super().__init__(process, wps_request, wps_response)
    self.port = self._assign_port()
    self.client = docker.from_env()
    self.cntnr = self._create()
\end{lstlisting}

\paragraph{\textit{self.\_assign\_port()}} The method returns the number of available port. The port is chosen from range <\textit{port\_min},
\textit{port\_max}> which are both configurable values. If no port from the range is available, the method returns
\textit{NoAvailablePortException}. Schema of ports assignment to each container is in Fig. \ref{fig:Diag_port}.

\paragraph{\textit{docker.from\_env()}} The \textit{docker} is a Python library for the Docker Engine API. \textit{from\_env} method
returns an instance of \textit{DockerClient} class which is a client to communicate with the Docker daemon. The returned client is
configured from the same variables as the Docker command-line client.

\paragraph{\textit{self.\_create()}} The \textit{\_create} method reads following values at the beginning:
\begin{itemize}
\item \textit{cntnr\_img} - Name of the image the container will be created from. The name of the image must be the same as the tag
set by the \textit{-t} parameter in \textit{docker build} command when the image is built from Dockerfile.

\bigskip
\begin{lstlisting}[basicstyle=\small,caption={Docker build command}]
docker build -t image_name /path/to/dockerfile
\end{lstlisting}

\item \textit{prcs\_inp\_dir} - Path to process workdir from \textit{self.job.wps\_response.process.workdir}. It is a directory where the
inputs for the process are stored.
\item \textit{prcs\_out\_dir} - Path to server output directory where all outputs are stored. The path is taken from \textit{outputpath}
parameter of section \textit{server} in the configuration file.
\item \textit{dckr\_inp\_dir} - Path to input data directory of WPS instance running inside Docker container. It is taken from 
\textit{dckr\_inp\_dir} of \textit{processing} section.
\item \textit{dckr\_out\_dir} - Path to output directory of WPS instance running inside the container. It is taken from 
\textit{dckr\_out\_dir} of \textit{processing} section.
\end{itemize}

The method returns an instance of \textit{Container} class from \textit{docker} module. The container is created by
\textit{self.client.containers.create()} method. 

The method takes optional parameter \textit{ports}. It is a dictionary to define ports to bind inside the container. The keys of the
dictionary are the ports to bind inside the container (port 5000 insinde \textit{Container 1} and \textit{Container 2} at Fig. \ref{fig:Diag_port}). The values of the dictionary are the corresponding ports to open on the host (port 5050 for \textit{Job1}, port 5051 for \textit{Job2} at Fig. \ref{fig:Diag_port}).

\begin{figure}[h!]
\centering
\includegraphics[width=0.55\textwidth]{img/Diag_ports.png}
\caption{Ports assignment schema}
\label{fig:Diag_port}
\end{figure}

Another optional parameter is \textit{volumes}. It is a dictionary 
to configure volumes mounted inside the container. The key is the host path and the value is a dictionary with the keys: \textit{bind}
- the path to mount the volume inside the container, and \textit{mode} - either \textit{rw} to mount the volume read/write, or 
\textit{ro} to mount it read-only.

\bigskip
\begin{lstlisting}[basicstyle=\small,caption={\textit{create()} method}]
self.client.containers.create(cntnr_img, detach=True,
ports={"5000/tcp": self.port}, 
volumes={prcs_out_dir: {'bind': dckr_out_dir, 'mode': 'rw'},
         prcs_inp_dir: {'bind': dckr_inp_dir, 'mode': 'ro'}})
\end{lstlisting}

Every container created with defined parameters \textit{volumes} and \textit{ports} will have output directory on the host mounted into 
the container output directory as well as the process workdir at host machine mounted into container directory with data. Therefore, all
inputs downloaded to process workdir will be available for the container and all outputs produced after process execution will be stored
at host machine output directory. Displayed in Fig. \ref{fig:Diag_mount}.

\begin{figure}[h!]
\centering
\includegraphics[width=0.65\textwidth]{img/Diag_mount.png}
\caption{Schema of mounting directories}
\label{fig:Diag_mount}
\end{figure}

\subsection{\textit{Container.start()} method}
When a container is created the \textit{start()} method is called and the container is started.
In the time of finishing this thesis the method looks like at Lst.\ref{lst:Container_start}. In the current state is used the method
\textit{\_dirty\_clean()}. It assures that the container is removed after the successful execution, temporary workdir is cleaned and it also 
updates the process status in the database. Unfortunately, it causes that the process runs syncronously. To solve this problem is one of the
future goals.
 
\begin{lstlisting}[basicstyle=\small,caption={\textit{Container.create()} method},label={lst:Container_start}]
def start(self):
    self.cntnr.start()
    time.sleep(0.5)
    self._execute()
    self._parse_status()
    self._dirty_clean()
\end{lstlisting}

\subsubsection{\textit{docker.container.start()}}
\textit{start()} method of \textit{Container} class from Python module \textit{docker}. The method is similar to \textit{docker start} command. 
It starts the Docker container. Then the method \textit{time.sleep()} is called to wait half a second after which the container is ready to 
use.

\subsubsection{\textit{Container.\_execute()}}
\textit{\_execute()} method handles forwarding execution from server to container. For sending request to container \textit{OWSLib} 
library\footnote{\url{https://github.com/geopython/OWSLib}} is used. 
\todo[inline]{Add appendix reference}

\paragraph{OWSLib} is a Python package for client programming with OGC web services interface standards. However before it was possible
to use this package there was necessary to fix a bug in the \textit{wps} module. The bug caused outputs in the Execute response, which
were referenced as web-accesible resource, not to be parsed because of wrong handling with \textit{xlink} namespace. The bug-fix diff file 
is available at App. 

\begin{lstlisting}[basicstyle=\small,caption={\textit{Container.\_execute()} method},label={lst:Container._execute}]
def _execute(self):
    url_execute = "http://localhost:{}/wps".format(self.port)
    inputs = get_inputs(self.job.wps_request.inputs)
    output = get_output(self.job.wps_request.outputs)
    wps = WPS(url=url_execute, skip_caps=True)
    self.execution = wps.execute(self.job.wps_request.identifier,
                                 inputs=inputs, output=output)
\end{lstlisting}

The method calls \textit{get\_inputs()} that 
returns all inputs transformed into a list of tuples in form \textit{(input\_name, input\_value)}. In case of ComplexData input, the input value is replaced with path to file. It is necessary to transform the inputs into the list of tuples because it is the required form for
\textit{WebProcessingService.execute()} method. Example for demo process \textit{buffer} at Lst. \ref{lst:Container.get_inputs}.

\begin{lstlisting}[basicstyle=\small,caption={\textit{get\_inputs} return value},label={lst:Container.get_inputs}]
the_inputs = [('poly_in', 'file:///pywps-flask/data/point.gml'),
              ('buffer', '1.0')]
the_outputs = [('buff_out', 'true')]
\end{lstlisting}

Then the method calls \textit{get\_outputs()} that returns list of tuples in form \textit{(output\_name, asReference\_attr\_value)}. It is 
necessary to transform the outputs into the list of tuples because it is the required form for \textit{WebProcessingService.execute()} method. 

\textit{WebProcessingService} object from OWSLib package is responsible for sending request to container. Its constructor takes URL of the
WPS server running inside container. Container URL varies depending on the port assigned to the container. Then the \textit{WPSExecution}
object is assigned to the \textit{Container} instance. The WPSExecution object is returned from \textit{WebProcessingService.execute()} 
method that takes as inputs process identifier, list of inputs from \textit{get\_inputs()} and outputs from \textit{get\_outputs()}.

\subsubsection{\textit{Container.\_parse\_status()}}
The method takes path to status location from \textit{WPSExecution.statusLocation} and copies it to \textit{Job.process.status\_url}.
Then the \textit{WPSReponse} object is updated by \textit{Job.wps\_response.update\_status()} with \textit{WPSExecution.statusMessage}.
It means the \textit{WPSResponse} object at host machine WPS server adopts \textit{statusMessage} and path to \textit{statusLocation}
from \textit{WPSExecution} object that handles the process execution inside the container. The process execution inside the container
updates its status into the file that is located in container output directory. This directory is shared with WPS server at host machine
si it is available even for the client.

\subsubsection{\textit{Container.\_dirty\_clean()}}
The method cares about stopping and removing Docker container, removing job workdir and original status XML. This method prevents to
accumulation of running Docker containers and temporary files in workdir directory. On the other hand there is missing functionality
for the process managment in database. In the current state when using Docker, the processes on the server are not ended even though
the result is already returned from the container. These pseudo-running processes accumulate on the server and some other processes
can be rejected because the limit of maximal running  processes is reached. This must be solved in the future.

\begin{lstlisting}[basicstyle=\small,caption={\textit{get\_inputs} return value},label={lst:Container.get_inputs}]
def _dirty_clean(self):
    time.sleep(1)
    self.cntnr.stop()
    self.cntnr.remove()
    self.job.process.clean()
    os.remove(self.job.process.status_location)
\end{lstlisting}

\textit{time.sleep(1)} is called to wait one second so the running process execution inside the container can be finished. The parameter
1 second is hardcoded and serves just now when the development is not done. \textit{stop()} and \textit{remove()} methods of class 
\textit{Container} from \textit{docker} module are similar to docker commands \textit{docker stop container\_id} and 
\textit{docker rm container\_id}.

\textit{Job.process.clean()} remove the job workdir so the temporary files do not cumulate at the server. \textit{os.remove()} deletes
the original status XML since the status XML from the container was sent back to the client.

\newpage
\necislovana{Závěr}
\todo[inline]{Dopsat zaver}

\newpage
\necislovana{Seznam použitých zkratek}

\begin{tabular}{ll}
\textbf{API}& Application Programming Interface\\
\textbf{CGAL}& Computational Geometry Algorithms Library\\
\textbf{GDAL}& Geospatial Data Abstraction Library\\
\textbf{GIS}& Geographic Information System\\
\textbf{HPC}& High Performance Compute\\
\textbf{KVP}& Key Value Pair\\
\textbf{OGC}& Open Geospatial Consortium\\
\textbf{PID}& Process identifier \\
\textbf{SOAP}& Simple Object Access Protocol \\
\textbf{URL}& Uniform Resource Locator\\
\textbf{VM}& Virtual Machine\\
\textbf{VMM}& Virtual Machine Monitor\\
\textbf{WPS}& Web Processing Service\\
\textbf{WMS}& Web Map Service\\
\textbf{WFS}& Web Feature Service\\
\textbf{WCS}& Web Coverage Service\\
\textbf{XML}& eXtensible Markup Language\\
\end{tabular}

\newpage
\begin{thebibliography}{99}
\label{Bibliography}
\bibitem{OGC_news}
Mark Reichardt \textit{OGC Newsletter - October 2004, OGC document number 04-043} [online].
URL: \textless\url{http://www.opengeospatial.org/pressroom/newsletters/200410}\textgreater

\bibitem{WPS_experiment}
Sam Bacharach \textit{OGC announces Web Processing Services Interoperability Experiment} [online].
URL: \textless\url{http://www.opengeospatial.org/pressroom/pressreleases/414}\textgreater

\bibitem{WPS_first}
Open Geospatial Consortium Inc. \textit{OpenGIS ® Web Processing Service, OGC document number 05-007r4, ver. 0.4.0} [online].
URL: \textless\url{https://portal.opengeospatial.org/files/?artifact_id=13149&version=1&format=doc}\textgreater

\bibitem{OGC}
Open Geospatil Consorcium \textit{OGC websites} [online].
URL: \textless\url{http://www.opengeospatial.org/}\textgreater

\bibitem{OGC_history}
Open Geospatil Consorcium \textit{OGC History (detailed)} [online].
URL: \textless\url{http://www.opengeospatial.org/ogc/historylong}\textgreater

\bibitem{WPS_standart_1.0}
http://www.opengeospatial.org/pressroom/newsletters/200410

\bibitem{WPS_second}
Open Geospatial Consortium \textit{OGC® WPS 2.0 Interface Standard Corrigendum 1, OGC document number 06-121r3} [online].
URL: \textless\url{https://portal.opengeospatial.org/files/?artifact_id=13149&version=1&format=doc}\textgreater

\bibitem{OGC_common}
Open Geospatial Consortium Inc. \textit{OGC Web Services Common Specification, OGC document number 14-065} [online].
URL: \textless\url{https://portal.opengeospatial.org/files/?artifact_id=20040}\textgreater

\bibitem{deegree_docs}
deegree devs \textit{deegree documentation} [online].
URL: \textless\url{http://download.deegree.org/documentation/3.3.20/html/}\textgreater

\bibitem{GS_docs}
GeoServer devs \textit{GeoServer documentation} [online].
URL: \textless\url{http://docs.geoserver.org/}\textgreater

\bibitem{ZOO_docs}
ZOO-Project devs \textit{ZOO-Project documentation} [online].
URL: \textless\url{http://zoo-project.org/docs}\textgreater

\bibitem{AG_docs}
Esri \textit{Tuning and configuring services} [online].
URL: \textless\url{http://server.arcgis.com/en/server/latest/publish-services/linux/tuning-and-configuring-services.htm}\textgreater

\bibitem{Docker_docs}
Docker \textit{Docker documentation} [online].
URL: \textless\url{https://docs.docker.com/}\textgreater

\bibitem{PyWPS_paper}
Jáchym Čepický, Luís Moreira de Sousa \textit{New implementation of OGC Web Processing Service in Python programming language.} [online].
URL: \textless\url{https://www.int-arch-photogramm-remote-sens-spatial-inf-sci.net/XLI-B7/927/2016/isprs-archives-XLI-B7-927-2016.pdf}\textgreater

\bibitem{PyWPS_presentation}
Jorge de Jesus, Luca Casagrande, Jáchym Čepický \textit{PyWPS a tutorial for beginners and developers} [online].
URL: \textless\url{https://www.slideshare.net/JorgeMendesdeJesus/pywps-a-tutorial-for-beginners-and-developers}\textgreater

\bibitem{PyWPS_web}
PyWPS developers \textit{PyWPS History} [online].
URL: \textless\url{http://pywps.org/history/}\textgreater

\bibitem{PyWPS_docs}
PyWPS developers \textit{PyWPS documentation} [online].
URL: \textless\url{http://pywps.readthedocs.io/}\textgreater

\bibitem{PyDev_ML}
Victor Stinner \textit{The pysandbox project is broken} [online].
URL: \textless\url{https://lwn.net/Articles/574323/}\textgreater

\bibitem{Korea}
Victor Stinner \textit{Linkage of OGC WPS 2.0 to the e-Government Standard Framework in Korea: An Implementation Case for Geo-Spatial Image Processing} [online].
URL: \textless\url{http://www.mdpi.com/2220-9964/6/1/25/pdf}\textgreater
\end{thebibliography}

\appendix

\newpage
\section{Dockerfile}
\begin{lstlisting}[basicstyle=\small,caption={Dockerfile example}]
ARG VERSION=0.9.22
FROM phusion/baseimage:${VERSION}
LABEL maintainer="devs@pywps.example.net"

RUN apt-get update -y && apt-get install -y \
	git \
	python3 \
	python3-dev

RUN git clone https://github.com/geopython/pywps-flask.git
 
WORKDIR /pywps-flask
RUN pip3 install -r requirements.txt

RUN mkdir /etc/service/pywps4
COPY pywps4_service.sh /etc/service/pywps4/run
RUN chmod +x /etc/service/pywps4/run

EXPOSE 5000
ENTRYPOINT ["/usr/bin/python3", "demo.py","-a"]
\end{lstlisting}

\newpage
\section{Execute request example}
\label{app:ExecuteRequest}
\begin{lstlisting}[basicstyle=\small,caption={Execute request example},language=XML]
<?xml version="1.0" encoding="UTF-8" standalone="yes"?>
<wps:Execute service="WPS" version="1.0.0" xmlns:wps="http://www.opengis.net/wps/1.0.0" xmlns:ows="http://www.opengis.net/ows/1.1" xmlns:xlink="http://www.w3.org/1999/xlink" xmlns:xsi="http://www.w3.org/2001/XMLSchema-instance" xsi:schemaLocation="http://www.opengis.net/wps/1.0.0 ../wpsExecute_request.xsd">
 <ows:Identifier>buffer</ows:Identifier>
 <wps:DataInputs>
  <wps:Input>
   <ows:Identifier>poly_in</ows:Identifier>
   <wps:Reference xlink:href="http://localhost:5000/static/data/point.gml" />
  </wps:Input>
  <wps:Input>
   <ows:Identifier>buffer</ows:Identifier>
   <wps:Data>
    <wps:LiteralData>1</wps:LiteralData>
   </wps:Data>
  </wps:Input>
 </wps:DataInputs>
 <wps:ResponseForm>
  <wps:ResponseDocument status="true" storeExecuteRsponse="true">
   <wps:Output asReference="true">
    <ows:Identifier>buff_out</ows:Identifier>
   </wps:Output>
  </wps:ResponseDocument>
 </wps:ResponseForm>
</wps:Execute>
\end{lstlisting}

\newpage
\section{List of tables and figures}
\listoftables

\listoffigures

\newpage
\section{ZIP file content}

\setlength{\unitlength}{.5mm}
\begin{picture}(250, 220)
  \put(  0, 212){\textbf{.}}
  \put(  1, 200){\line(0, 1){5}}
  \put(  1, 200){\line(1, 0){10} {\textbf{ text}}}
  \put( 16, 190){\line(0, 1){8}}
  \put( 16, 190){\line(1, 0){10} {\textbf{ LaTeX}}}
  \put(150, 190){ zdrojové soubory textu}
  \put( 16, 180){\line(0, 1){10}}
  \put( 16, 180){\line(1, 0){10} { adam-laza-bp-2015.pdf}}
  \put(150, 180){ tento text}
  \put( 16, 170){\line(0, 1){10}}
  \put( 16, 170){\line(1, 0){10} { adam-laza-bp-2015.xls}}
  \put(150, 170){ anotace práce}
  \put( 16, 160){\line(0, 1){10}}
  \put( 16, 160){\line(1, 0){10} { zadani.pdf}}
  \put(150, 160){ naskenované oficiální zadání této práce}
  \put(  1, 150){\line(0, 1){60}}
  \put(  1, 150){\line(1, 0){10} {\textbf{ src}}}
  \put( 16, 140){\line(0, 1){8}}
  \put( 16, 140){\line(1, 0){10} {\textbf{ nnbathy}}}
  \put(150, 140){ zdrojové kódy nnbathy}
  \put( 16, 130){\line(0, 1){10}}
  \put( 16, 130){\line(1, 0){10} {\textbf{ cgal}}}
  \put(150, 130){ zdrojové kódy cgal}
  \put(  1, 120){\line(0, 1){50}}
  \put(  1, 120){\line(1, 0){10} {\textbf{ testing}}}
  \put( 16, 110){\line(0, 1){8}}
  \put( 16, 110){\line(1, 0){10} {\textbf{ scripts}}}
  \put(150, 110){ testovací skripty}
  \put( 16, 100){\line(0, 1){10}}
  \put( 16, 100){\line(1, 0){10} {\textbf{ sample\_data}}}
  \put(150, 100){ testovací data}
\end{picture}

\end{document}
