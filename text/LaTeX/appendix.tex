\newpage
\part{Appendix}

\appendix
\newpage
\section{Execute request example}
\label{app:ExecuteRequest}
\begin{lstlisting}[basicstyle=\small,caption={Execute request example}]
<?xml version="1.0" encoding="UTF-8" standalone="yes"?>
<wps:Execute service="WPS" version="1.0.0" xmlns:wps="http://www.opengis.net/wps/1.0.0" xmlns:ows="http://www.opengis.net/ows/1.1" xmlns:xlink="http://www.w3.org/1999/xlink" xmlns:xsi="http://www.w3.org/2001/XMLSchema-instance" xsi:schemaLocation="http://www.opengis.net/wps/1.0.0 ../wpsExecute_request.xsd">
 <ows:Identifier>buffer</ows:Identifier>
 <wps:DataInputs>
  <wps:Input>
   <ows:Identifier>poly_in</ows:Identifier>
   <wps:Reference xlink:href="http://localhost:5000/static/data/point.gml" />
  </wps:Input>
  <wps:Input>
   <ows:Identifier>buffer</ows:Identifier>
   <wps:Data>
    <wps:LiteralData>1</wps:LiteralData>
   </wps:Data>
  </wps:Input>
 </wps:DataInputs>
 <wps:ResponseForm>
  <wps:ResponseDocument status="true" storeExecuteRsponse="true">
   <wps:Output asReference="true">
    <ows:Identifier>buff_out</ows:Identifier>
   </wps:Output>
  </wps:ResponseDocument>
 </wps:ResponseForm>
</wps:Execute>
\end{lstlisting}

\newpage
\section{Execute response example (async mode)}
\label{app:ExecuteResponse}
\begin{lstlisting}[basicstyle=\small,caption={Execute response example (async mode)}]
<!-- PyWPS 4.0.0 -->
<wps:ExecuteResponse xmlns:gml="http://www.opengis.net/gml" 
  xmlns:ows="http://www.opengis.net/ows/1.1" 
  xmlns:wps="http://www.opengis.net/wps/1.0.0" 
  xmlns:xlink="http://www.w3.org/1999/xlink" 
  xmlns:xsi="http://www.w3.org/2001/XMLSchema-instance" 
  xsi:schemaLocation="http://www.opengis.net/wps/1.0.0 
  http://schemas.opengis.net/wps/1.0.0/wpsExecute_response.xsd" 
  service="WPS" version="1.0.0" xml:lang="en-US" 
  serviceInstance="http://localhost:5000/wps?service=WPS&request=
  GetCapabilities" 
  statusLocation="http://localhost:5000/outputs/ce57acbe-f1f3-11
  e7-ad2a-0242ac110003.xml">
 <wps:Process wps:processVersion="0.1">
  <ows:Identifier>buffer</ows:Identifier>
  <ows:Title>GDAL Buffer process</ows:Title>
  <ows:Abstract>
   The process returns buffers around the input features,
   using the GDAL library
  </ows:Abstract>
 </wps:Process>
 <wps:Status creationTime="2018-01-05T09:38:41Z">
  <wps:ProcessAccepted>
   PyWPS Process buffer accepted
  </wps:ProcessAccepted>
 </wps:Status>
</wps:ExecuteResponse
\end{lstlisting}

\newpage
\section{Status XML example with referenced output}
\label{app:status_reference}
\begin{lstlisting}[basicstyle=\small,caption={Status XML example}]
<wps:ExecuteResponse
  xsi:schemaLocation="http://www.opengis.net/wps/1.0.0 
  http://schemas.opengis.net/wps/1.0.0/wpsExecute_response.xsd" 
  service="WPS" version="1.0.0" xml:lang="en-US"
  serviceInstance="http://localhost:5000/wps?service=WPS&request=
  GetCapabilities" 
  statusLocation="http://localhost:5000/outputs/ce57acbe-f1f3-11
  e7-ad2a-0242ac110003.xml">
 <wps:Process wps:processVersion="0.1">
  <ows:Identifier>buffer</ows:Identifier>
  <ows:Title>GDAL Buffer process</ows:Title>
  <ows:Abstract>
   The process returns buffers around the input features,
   using the GDAL library
  </ows:Abstract>
 </wps:Process>
 <wps:Status creationTime="2018-01-05T08:38:30Z">
  <wps:ProcessSucceeded>
   PyWPS Process GDAL Buffer process finished
  </wps:ProcessSucceeded>
 </wps:Status>
 <wps:ProcessOutputs>
  <wps:Output>
   <ows:Identifier>buff_out</ows:Identifier>
   <ows:Title>Buffered file</ows:Title>
   <wps:Reference xlink:href="http://localhost:5000/outputs/ce57acbe-f1f3-11e7-ad2a-0242ac110003/point_buffer_42rkmvt1.gml" mimeType="application/gml+xml"/>
  </wps:Output>
 </wps:ProcessOutputs>
</wps:ExecuteResponse>
\end{lstlisting}

\section{Status XML example with inline output}
\label{app:status_inline}
\begin{lstlisting}[basicstyle=\small,caption={Status XML example}]
<wps:ExecuteResponse
  xsi:schemaLocation="http://www.opengis.net/wps/1.0.0 
  http://schemas.opengis.net/wps/1.0.0/wpsExecute_response.xsd" 
  service="WPS" version="1.0.0" xml:lang="en-US" 
  serviceInstance="http://localhost:5000/wps?service=WPS&request=
  GetCapabilities" 
  statusLocation="http://localhost:5000/outputs/1cd3e506-f1f7-11
  e7-8546-0242ac110003.xml">
 <wps:Process wps:processVersion="0.1">
  <ows:Identifier>buffer</ows:Identifier>
  <ows:Title>GDAL Buffer process</ows:Title>
  <ows:Abstract>
   The process returns buffers around the input features,
   using the GDAL library
  </ows:Abstract>
 </wps:Process>
 <wps:Status creationTime="2018-01-05T09:02:10Z">
  <wps:ProcessSucceeded>
   PyWPS Process GDAL Buffer process finished
  </wps:ProcessSucceeded>
 </wps:Status>
 <wps:ProcessOutputs>
  <wps:Output>
   <ows:Identifier>buff_out</ows:Identifier>
   <ows:Title>Buffered file</ows:Title>
   <wps:Data>
    <wps:ComplexData mimeType="application/gml+xml">
     <ogr:FeatureCollection xmlns:ogr="http://ogr.maptools.org/" 
       xsi:schemaLocation="http://schemas.opengis.net/gml/2.1.2/
       feature.xsd">
      <gml:boundedBy>
       <gml:Box>
        <gml:coord>
         <gml:X>-0.9514645979959721</gml:X>
         <gml:Y>-0.986306232731747</gml:Y>
        </gml:coord>
        <gml:coord>
         <gml:X>1.048535402004028</gml:X>
         <gml:Y>1.013693767268253</gml:Y>
        </gml:coord>
       </gml:Box>
      </gml:boundedBy>
      <gml:featureMember>
      <ogr:point_buffer fid="point_buffer.0">
       <ogr:geometryProperty>
        <gml:Polygon>
         <gml:outerBoundaryIs>
          <gml:LinearRing>
           <gml:coordinates>1.04853540200403,0.013693767268253 
           1.0471649367586,-0.038642188974691  0.857552396378976,
           -0.57409148502422 0.825681363460999,-0.615626623781584 
           0.791680227481423,-0.655436839090605 0.75564218319056,
           -0.852331636516187 -0.496103633010996,-0.8249768006773
           -0.539249850288442,-0.795323227106697 -0.580784989006,
           -0.763452194188721 -0.620595204354827,-0.7294510582044
           </gml:LinearRing>
          </gml:outerBoundaryIs>
         </gml:Polygon>
        </ogr:geometryProperty>
       </ogr:point_buffer>
      </gml:featureMember>
     </ogr:FeatureCollection>
    </wps:ComplexData>
   </wps:Data>
  </wps:Output>
 </wps:ProcessOutputs>
</wps:ExecuteResponse>
\end{lstlisting}

\newpage
\section{Dockerfile}
\label{app:dockerfile}
\begin{lstlisting}[basicstyle=\small,caption={Dockerfile example}]
FROM alpine:latest
MAINTAINER Jorge S. Mendes de Jesus <jorge.dejesus@geocat.net>

ENV GDAL_VERSION 2.2.0
ENV XERCES_VERSION 3.2.0

RUN apk add --no-cache \
	git \
	gcc \
	bash \
	openssh \
	musl-dev  \
	python3 \
	python3-dev \
	libxml2-dev  \
	libxslt-dev \
	linux-headers \
	expat \
	expat-dev


RUN apk --update --no-cache add g++ libstdc++ make swig

# Xerces
RUN wget http://www.apache.org/dist/xerces/c/3/sources/xerces-c-${XERCES_VERSION}.tar.gz -O /tmp/xerces-c-${XERCES_VERSION}.tar.gz && \
    tar xvf /tmp/xerces-c-${XERCES_VERSION}.tar.gz -C /tmp && \
    cd /tmp/xerces-c-${XERCES_VERSION} && \
    ./configure --prefix=/opt/xerces && \
    make -j 4 && \
    make install

# Geos
RUN apk add --no-cache \
    --repository http://dl-cdn.alpinelinux.org/alpine/edge/testing \
    geos \
    geos-dev

# Install GDAL
RUN wget http://download.osgeo.org/gdal/${GDAL_VERSION}/gdal-${GDAL_VERSION}.tar.gz -O /tmp/gdal.tar.gz && \
	tar xzf /tmp/gdal.tar.gz -C /tmp && \
	cd /tmp/gdal-${GDAL_VERSION} && \
	CFLAGS="-g -Wall" LDFLAGS="-s" ./configure --with-expat=yes --with-xerces=/opt/xerces --with-geos=yes \
	&& make -j 4 && make install

RUN cd /tmp/gdal-${GDAL_VERSION}/swig/python \
	&& python3 setup.py build \
	&& python3 setup.py install

RUN git clone https://github.com/lazaa32/pywps-flask.git

WORKDIR /pywps-flask
RUN pip3 install -r requirements.txt


EXPOSE 5000
ENTRYPOINT ["/usr/bin/python3", "demo.py","-a"]

#docker build -t pywps-flask .
#docker run -p 5000:5000 pywps-flask
#http://localhost:5000/wps?request=GetCapabilities&service=WPS
#http://localhost:5000/wps?request=DescribeProcess&service=WPS&identifier=all&version=1.0.0
\end{lstlisting}

\newpage
\section{OWSLib diff file}
\label{app:owslib}
\begin{lstlisting}[basicstyle=\small,caption={OWSLib diff file}]
diff --git a/owslib/wps.py b/owslib/wps.py
index c16e288..ce86f93 100644
--- a/owslib/wps.py
+++ b/owslib/wps.py
@@ -1117,13 +1117,15 @@ class Output(InputOutput):
 
   # extract wps namespace from outputElement itself
   wpsns = getNamespace(outputElement)
+  # extract xlink namespace
+  xlins =outputElement.nsmap['xlink']
 
   # <ns:Reference encoding="UTF-8" mimeType="text/csv"
   # href="http://cida.usgs.gov/climate/gdp/process/RetrieveResultServlet?id=1318528582026OUTPUT.601bb3d0-547f-4eab-8642-7c7d2834459e"
   # />
   referenceElement = outputElement.find(nspath('Reference', ns=wpsns))
   if referenceElement is not None:
-      self.reference = referenceElement.get('href')
+      self.reference = referenceElement.get('{{{}}}href'.format(xlins))
       self.mimeType = referenceElement.get('mimeType')
 
   # <LiteralOutput>
\end{lstlisting}

\newpage
\section{PyWPS-demo diff file (shortened)}
\label{app:pywps-demo}
\begin{lstlisting}[basicstyle=\small,caption={pywps-demo diff file}]
diff --git a/pywps.cfg b/pywps.cfg
index a1ed125..f6e3981 100644
--- a/pywps.cfg
+++ b/pywps.cfg
@@ -28,14 +28,25 @@ maxrequestsize=3mb
 url=http://localhost:5000/wps
 outputurl=http://localhost:5000/outputs/
 outputpath=outputs
-workdir=/tmp
+workdir=workdir
+wd_inp_subdir=inputs
+wd_out_subdir=outputs
 maxprocesses=10
-parallelprocesses=2
+parallelprocesses=6
+
+[processing]
+mode=multiprocessing
+port_min=5050
+port_max=5070
+docker_img=pywps_container
+dckr_inp_dir=/pywps-flask/data
+dckr_out_dir=/pywps-flask/outputs
 
 [logging]
 level=INFO
 file=logs/pywps.log
 database=sqlite:///logs/pywps-logs.sqlite3
+format=%(asctime)s] [%(levelname)s] file=%(pathname)s line=%(lineno)s module=%(module)s function=%(funcName)s %(message)s
\end{lstlisting}

\newpage
\section{PyWPS diff file (shortened)}
\label{app:pywps}
\begin{lstlisting}[basicstyle=\small,caption={pywps diff file}]
diff --git a/pywps/exceptions.py b/pywps/exceptions.py
index 8483911..e0e1c57 100644
--- a/pywps/exceptions.py
+++ b/pywps/exceptions.py
@@ -150,3 +150,10 @@ class SchedulerNotAvailable(NoApplicableCode):
     """Job scheduler not available exception implementation
     """
     code = 400
+
+
+class NoAvailablePortException(NoApplicableCode):
+    """
+    No port available for new docker.
+    """
+    code = 400
diff --git a/pywps/processing/__init__.py b/pywps/processing/__init__.py
index 03bf0af..63816e1 100644
--- a/pywps/processing/__init__.py
+++ b/pywps/processing/__init__.py
@@ -7,6 +7,7 @@
 import pywps.configuration as config
 from pywps.processing.basic import MultiProcessing
 from pywps.processing.scheduler import Scheduler
+from pywps.processing.container import Container
 # api only
 from pywps.processing.basic import Processing  # noqa: F401
 from pywps.processing.job import Job  # noqa: F401
@@ -16,6 +17,7 @@ LOGGER = logging.getLogger("PYWPS")
 
 MULTIPROCESSING = 'multiprocessing'
 SCHEDULER = 'scheduler'
+DOCKER = 'docker'
 DEFAULT = MULTIPROCESSING
 
 
@@ -30,6 +32,8 @@ def Process(process, wps_request, wps_response):
     LOGGER.info("Processing mode: %s", mode)
     if mode == SCHEDULER:
         process = Scheduler(process, wps_request, wps_response)
+    elif mode == DOCKER:
+        process = Container(process, wps_request, wps_response)
     else:
         process = MultiProcessing(process, wps_request, wps_response)
     return process
diff --git a/pywps/processing/container.py b/pywps/processing/container.py
new file mode 100644
index 0000000..8f5151e
--- /dev/null
+++ b/pywps/processing/container.py
@@ -0,0 +1,155 @@
+class ClientError:
+    pass
+
+class Container(Processing):
+    def __init__(self, process, wps_request, wps_response):
+    def _create(self):
+    def _assign_port(self):
+    def start(self):
+    def stop(self):
+    def cancel(self):
+    def pause(self):
+    def unpause(self):
+    def _execute(self):
+    def _parse_outputs(self):
+    def _parse_status(self):
+    def _dirty_clean(self):
+def get_inputs(job_inputs):
+def get_output(job_output):

diff --git a/pywps/response/execute.py b/pywps/response/execute.py
index f78cfb0..e994de3 100644
--- a/pywps/response/execute.py
+++ b/pywps/response/execute.py
@@ -15,7 +15,7 @@ from pywps import WPS, OWS
-import pywps.dblog
+from pywps.dblog import update_response
 
@@ -38,6 +38,33 @@ class ExecuteResponse(WPSResponse):
         self.process = kwargs["process"]
         self.outputs = {o.identifier: o for o in self.process.outputs}
 
+    def update_status(self, message=None, status_percentage=None, status=None,
+                      clean=True):
\end{lstlisting}

\newpage
\section{Docker extension documentation (shortened)}
\label{app:docker_ext_docs}
\begin{lstlisting}[basicstyle=\small,caption={Docker extension documentation}]
Docker Container Extension
==========================
To isolate each process execution it is possible to enable docker extension.
.. note:: The PyWPS process implementations are not changed by using the
  scheduler extension.
First of all install Docker from `website <https://docs.docker.com/engine/installation/linux/docker-ce/ubuntu/>`_.
Clone ``pywps-demo``::
  $ git clone https://github.com/lazaa32/pywps-flask.git
Install demo requirements from ``requirement.txt``. It will download all required packages including ``pywps`` core package::
  $ cd pywps-flask
  $ pip install -r requirements.txt

``pywps`` package was downloaded to ``src`` directory. Let's set the ``PYTHONPATH`` so ``pywps-demo`` knows where to find::
  $ EXPORT PYTHONPATH=$PYTHONPATH:$PWD/src/pywps-develop
If everything went OK, it should be now possible to run::
  $ python3 demo.py
However the demo still runs without Docker extension. First of all it is necessary to build an image from Dockerfile.
From the image all containers will be created::
  $ cd docker/isolation
  $ docker build -t container .
.. note:: The **-t** flag sets a name and optionally a tag in the **name:tag** format. The name of the image
   will be one of the parameter value in configuration file.
.. warning:: The image build can take up to several tens of minutes since some manual installation run on the
   background.
You can check the image was built by::
  $ docker images
To activate this extension you need to edit the ``pywps.cfg`` configuration file and make the following changes::
  [processing]
  mode=docker
  port_min=5050
  port_max=5070
  docker_img=container
  dckr_inp_dir=/pywps-flask/data
  dckr_out_dir=/pywps-flask/outputs
``mode`` must be set to ``docker``. ``port_min`` and ``port_max`` define the range of ports which can be
assigned to containers. ``docker_img`` must match to name of the image given by -t flag during the image build.
The docker extension is now enabled and every asynchronous request will be executed separately in a Docker
container.
\end{lstlisting}

\newpage
\section{List of tables and figures}
\listoftables

\listoffigures

\newpage
\section{ZIP file content}

\setlength{\unitlength}{.5mm}
\begin{picture}(250, 220)
  \put(  0, 212){\textbf{.}}
  \put(  1, 200){\line(0, 1){5}}
  \put(  1, 190){\line(0, 1){10}}
  \put(  1, 190){\line(1, 0){10} {\textbf{ src}}}
  \put( 16, 180){\line(0, 1){8}}
  \put( 16, 180){\line(1, 0){10} {\textbf{ diffs}}}
  \put( 31, 170){\line(0, 1){8}}
  \put( 31, 170){\line(1, 0){10} { owslib.diff}}
  \put(150, 170){owslib diff file}
  \put( 31, 160){\line(0, 1){10}}
  \put( 31, 160){\line(1, 0){10} { pywps.diff}}
  \put(150, 160){pywps diff file}
  \put( 31, 150){\line(0, 1){10}}
  \put( 31, 150){\line(1, 0){10} { pywps-demo.diff}}
  \put(150, 150){pywps-demo diff file}
  \put(  1, 140){\line(0, 1){50}}
  \put(  1, 140){\line(1, 0){5} {\textbf{ text}}}
  \put( 16, 130){\line(0, 1){8}}
  \put( 16, 130){\line(1, 0){10} {\textbf{ LaTeX}}}
  \put(150, 130){ LaTeX source code}
  \put( 16, 120){\line(0, 1){10}}
  \put( 16, 120){\line(1, 0){10} { adam-laza-bp-2015.pdf}}
  \put(150, 120){ text of the thesis}
  \put(  1, 110){\line(0, 1){50}} 
  \put(  1, 110){\line(1, 0){10} {\textbf{ zadani}}} 
  \put( 16, 100){\line(0, 1){8}}
  \put( 16, 100){\line(1, 0){10} { zadanidp.pdf}} 
  \put(150, 100){assignment of the thesis}

\end{picture}
