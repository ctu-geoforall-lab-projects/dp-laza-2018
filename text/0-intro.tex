\necislovana{Introduction} With the huge progress of technologies, our
society is becoming more and more digitalized and the amount of
various data is getting bigger and bigger. There are data all around
us and demand for applications or services based on the data is
growing. However, the data in its raw form may not be sufficient to
make a~conclusion. More often the data need to be processed and used
as inputs data for some kind of ana\-lyses. With increasing number of
gathered data such as satellite images or remotely sensed data, any
manual processing is almost inconceivable. The data processing needs
to be done in a systematic and fully-automized way.

Therefore, in order to be able to process data independently of the
type of acquisition, format or platform, international standard
interfaces and standardized frameworks are necessary. The Open
Geospatial Consortium, Inc.  (OGC) - an organization oriented toward
%% ML: to and tu ma byt? nejak mi ta veta prijde kostrbata...
%% AL: myslis: ''Geospatial Consortium, Inc.  (OGC), an organization oriented toward??''
%% ML: asi prepsano, uz to nemuzu najit
open geospatial standards - researches and establishes technical
standards for data compatibility and interoperability technical
standards. Besides quite famous and used standards as WMS and WFS,
there exists the WPS standard. The WPS standard defines an interface
that facilitates the publishing of geospatial processes. It provides
rules how inputs and outputs are handled. 
%% ML: ta veta s 'and' opet zni divne, moc mi nedava smysl, co zname 'only'?
%% AL: chtel jsem rict, ze WPS je jen standard, ze se za samotnym
%% WPS neni zadny kod. Ale prepsal jsem to
%% ML: OK
There are several implementations of WPS standard. This work is primarily focused
%% ML: chybi kontext na predchozi vetu, pywps je prave jedna z implementaci
%% AL: lepsi?
%% ML: OK
on the \textit{PyWPS} - a WPS implementation written in Python.

\bigskip
The main topic of this thesis is process isolation in PyWPS framework. A process is just some geospatial operation which 
has its defined inputs and outputs and which is deployed on a server. The server is able to execute multiple 
processes at the same time. This thesis deals with the isolation of individual processes, especially for security and 
performance reasons. With every process fully isolated, so they cannot interact with each other, the higher security 
level is ensured.

The thesis is composed of several parts. The first part describes the WPS standard, its operations 
\textit{GetCapabilities}, \textit{DescribeProcess} and \textit{Execute} and inputs and outputs structures. A quick
overview of some implementations of WPS standard follows and brings a basic information about them.

Nevertheless, this work is dedicated to PyWPS, an implementation in
Python. In the second part, its current state is described as well as
\textit{pywps-demo} - a side project providing demo server instance -
%% ML: and covers? opet divne formulovana veta
%% AL: prepsano. To cover - ve vyznamu pojednavat.
which the practical part is based on. Following research 
covers various projects and technologies which were considered as a
solution for process isolation. Eventually, the Docker technology is
chosen for the implementation part.  Docker has been selected as one
of the most used technology for containerization. It puts every
process into a separate container so the isolation is
ensured. Moreover, Docker provides a mechanism to pause, stop and
start a container so it looks like a possible solution for the future
WPS 2.0.0 standard implementation which requires this
functionality. Using Docker, it also opens new possibilities,
e.g. being able to deploy running job to cloud.

The third part describes the implementation. It explains the Execute
operation workflow, a process execution and how the Docker containers
are used for the PyWPS process isolation. New \textit{Container}
%% ML: were -> was?
%% AL: was
class, which was developed during the work on this thesis, is
introduced as well as its methods.
