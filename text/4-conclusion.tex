\newpage
\necislovana{Conclusion}

The goal of the thesis was to find and implement a solution for
process isolation in \textit{PyWPS}. This functionality was demanded
by PyWPS developers who would appreciate the possibility to isolate
each process execution. With every process fully isolated, a higher
level of security is ensured. Moreover, without the isolation the
processes have access to a file-system of a hosting OS.

But there are other reasons considered. One of them is a
performance. Non-isolated processes share the resources of a host
machine. In case that a client requests an execution of a process that
is poorly designed, its execution can consume a lot of resources and
thus it may slow down other process executions running in parallel. In
the worst-case scenario a process execution can bring down the server.

In the first part, the thesis sets a theoretical background and
%% ML: explains / is explained
%% AL: is explained
%% ML: Some/Various projects ...
%% AL: Various
%% ML: prepsal jsem project -> projects
the \textit{WPS} standard is explained. Various projects which implement the
standard are mentioned. Second part is dedicated to \textit{PyWPS} as
a Python implementation of the WPS. There is described the
currentstate of the project followed by a research.

%% ML: solutions?
%% AL: solutions
The reseach covers various solutions for the process isolation. The
functionality for the isolation was the main criterion, however beside
that the selected solution should provide some mechanisms to control
the execution of the process. These mechanisms will be necessary for
implementation of the WPS 2.0.0 standard.

The \textit{Docker} Container Extension has been on the developers
%% ML: wishlist ?
%% AL: whislist
%% ML: OK, uz opraveno v textu
wishlist for a long time. The container encapsulates the process
execution and also offers methods to start, pause, stop or kill the
container and thus the execution. Moreover using Docker opens
possibilities of Web Processing Service in a cloud computing
infrastructure.

The architecture is based on \textit{pywps-demo} project. It offers a
demo server instance of PyWPS. When a process execution is requested,
a server creates a Docker container with the demo server instance
running inside. The request is forwarded into the container. The
process execution runs inside the container but the container output
directory is mounted into the server file-system so the results are
available on the server.

\newpage
At the time of submitting, the implementation is not finished. It
provides working solution but there are several issues to be
considered. First of all, the solution does not provide an
asynchronous execution. A client receives an execute response that the
process was accepted, after the execution is already finished.

Another problem is a management of containers. In the current state, a
container is stopped and removed after a successful process execution,
however it is achieved with the \textit{\_dirty\_clean()} method which
is inappropriate for production environment.

%% ML: tomu nerozumim, ten proces na server zustane vyset i kdyz uz je
%% v kontejneru hotov a odbaven k uzivateli ?
%% AL: ano, to je problem, ktery budu muset jeste vyresit. Proces
%% dobehne v kontejneru, uzivatel si muze stahnout vysledky
%% ale process na lokalnim serveru se neukonci
%% ML: ops...
Another problem is cumulation of processes on a server. These
pseudo-running processes remain on the server and block other
processes even though a client has already received the results. There
is missing connection between a container and a server when a process
execution inside the container is done.

All these issues prevents from integration the Docker container
extension into the official PyWPS repository.  Nevertheless, I would
like to continue on the development and solve all problems so I can
make a pull request at least into PyWPS develop branch. During the
%% ML: nejsem si jisty zda uvadet linky na tyto repositare
%% AL: linky na repositare smazany
%% ML: urcite uved ale odkaz na repositar diplomky, tam budou stejne
%% vsechny diffy, to tady muzes zminit
%% AL: pridan link
implementation I have contributed into \textit{pywps}, \textit{pywps-demo}
and \textit{OWSLib} projects. Pull requests
into \textit{pywps} and \textit{pywps-demo} wait for the PyWPS
%% ML: uved do poznamky pod carou link
%% AL: link k cemu?
%% ML: k tomu pull requestu :-)
developers feedback and problems solution mentioned above. The pull
%% ML: zminujes, ze je pull request zalozen z ceka na schvaleni, muzes uvest jeho URL
request into OWSLib is already waiting for approval. All diff files,
as well the text of the thesis and its source code,
are available at GitHub repository of this 
thesis.\footnote{\url{https://github.com/ctu-geoforall-lab-projects/dp-laza-2018/}}
